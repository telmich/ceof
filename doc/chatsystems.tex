% ----------------------------------------------------------------------------
\chapter{Analysis and Comparision of Chat Systems}

    1. Detailed analysis and comparison of open and legacy chat systems
        to summarise current chat system features and their
        security characteristics.


% ----------------------------------------------------------------------------
\section{IRC}
\subsection{History}

Since 1989 deleoped\ref{rfc2810},
first formally documented in May 1993 by RFC 1459\ref{rfc1459}.

\begin{quote}
All client-to-server IRC protocols in use today are descended from the protocol implemented in the irc2.4.0 version of the IRC2 server, and documented in RFC 1459. Since RFC 1459 was published, the new features in the irc2.10 implementation led to the publication of several revised protocol documents (RFC 2810, RFC 2811, RFC 2812 and RFC 2813); however, these protocol changes have not been widely adopted among other implementations.\ref{irc-wp}
\end{quote}

% ----------------------------------------------------------------------------
\subsection{Architecture}
central server, server network
IRC, the \textit{Internet Relay Chat}, has been developed since 1989
and is defined in severals Internet RFCs.\footnote{See \cite{rfc1459}, 
\cite{rfc2810}, \cite{rfc2812} and \cite{rfc2813}.}
IRC networks consist of IRC servers and IRC clients:

\begin{verbatim}
                       1--\
                           A        D---4
                       2--/ \      /
                             B----C
                            /      \
                           3        E

   Servers: A, B, C, D, E         Clients: 1, 2, 3, 4

                    [ Fig. 1. Sample small IRC network ]

\end{verbatim}\footnote{Source: \cite{rfc2810}}
IRC is organised centrally, as stated in \cite{rfc2810}:
\begin{quote}
The IRC protocol provides no mean for two clients to directly
communicate.  All communication between clients is relayed by the
server(s).
\end{quote}

There is, however, an unstandartised client extension named 
\textit{Direct Client-to-Client} (DCC) available in most IRC clients
that enables direct connections.\footnote{See \cite{dcc}, \cite{dcc2}.}

% ----------------------------------------------------------------------------
\subsection{Security}
SSL is being used in some networks, but not standartised. [nico]
optional encryption.
% ----------------------------------------------------------------------------
\section{ICQ}
% ----------------------------------------------------------------------------
\subsection{History}
November 1996 veröffentlicht.
 the first Internet-wide instant messaging service, 

America Online acquired Mirabilis on June 8, 1998, for 407dollar million (dollar287 million in cash and dollar120 million over a three-year period based on growth performance levels).

In 2001, ICQ had over 100 million accounts registered.[2] In April 2010, AOL sold ICQ to Digital Sky Technologies for dollar187.5 million.[3]

% ----------------------------------------------------------------------------
\subsection{Architecture}
% ----------------------------------------------------------------------------
\section{Silc}
% ----------------------------------------------------------------------------
\subsection{History}
% ----------------------------------------------------------------------------
\subsection{Architecture}
% ----------------------------------------------------------------------------
\subsection{Security}

SILC Project develops the Secure Internet Live Conferencing protocol (SILC),
\url{http://silcnet.org/general/}
\url{http://silcnet.org/support/documentation/specs/}

central
\url{http://silcnet.org/support/documentation/wp/silc_protocol.php}


% ----------------------------------------------------------------------------
\section{Jabber}
% ----------------------------------------------------------------------------
\section{Skype}
More VoIP
% ----------------------------------------------------------------------------
\section{MSN}
More VoIP
% ----------------------------------------------------------------------------
\section{ICQ}
More VoIP


% ----------------------------------------------------------------------------
\section{Security features and Comparision}

\begin{longtable}{|c|c|c|}
\caption{Chat system comparision with security features}\\
\hline
\textbf{Name} & \textbf{IRC} & \textbf{SILC}\\
\hline
\textbf{Single point of attack} & yes & yes\\
\hline
\textbf{Encrypted traffic} & optional & yes\\
\hline
\end{longtable}

All the solutions with objectives.
