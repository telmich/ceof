% ----------------------------------------------------------------------------
\chapter{Analysis and Comparision of Chat Systems}

    1. Detailed analysis and comparison of open and legacy chat systems
        to summarise current chat system features and their
        security characteristics.


% ----------------------------------------------------------------------------
\section{IRC}
\subsection{History}

Since 1989 deleoped\ref{rfc2810},
first formally documented in May 1993 by RFC 1459\ref{rfc1459}.

\begin{quote}
All client-to-server IRC protocols in use today are descended from the protocol implemented in the irc2.4.0 version of the IRC2 server, and documented in RFC 1459. Since RFC 1459 was published, the new features in the irc2.10 implementation led to the publication of several revised protocol documents (RFC 2810, RFC 2811, RFC 2812 and RFC 2813); however, these protocol changes have not been widely adopted among other implementations.\ref{irc-wp}
\end{quote}

% ----------------------------------------------------------------------------
\subsection{Architecture}
central server, server network
IRC, the \textit{Internet Relay Chat}, has been developed since 1989
and is defined in severals Internet RFCs.\footnote{See \cite{rfc1459}, 
\cite{rfc2810}, \cite{rfc2812} and \cite{rfc2813}.}
IRC networks consist of IRC servers and IRC clients:

\begin{verbatim}
                       1--\
                           A        D---4
                       2--/ \      /
                             B----C
                            /      \
                           3        E

   Servers: A, B, C, D, E         Clients: 1, 2, 3, 4

                    [ Fig. 1. Sample small IRC network ]

\end{verbatim}\footnote{Source: \cite{rfc2810}}
IRC is organised centrally, as stated in \cite{rfc2810}:
\begin{quote}
The IRC protocol provides no mean for two clients to directly
communicate.  All communication between clients is relayed by the
server(s).
\end{quote}

There is, however, an unstandartised client extension named 
\textit{Direct Client-to-Client} (DCC) available in most IRC clients
that enables direct connections.\footnote{See \cite{dcc}, \cite{dcc2}.}

% ----------------------------------------------------------------------------
\subsection{Security}
SSL is being used in some networks, but not standartised. [nico]
optional encryption.
% ----------------------------------------------------------------------------
\section{ICQ}
% ----------------------------------------------------------------------------
\subsection{History}
November 1996 veröffentlicht.
 the first Internet-wide instant messaging service, 

America Online acquired Mirabilis on June 8, 1998, for 407dollar million (dollar287 million in cash and dollar120 million over a three-year period based on growth performance levels).

In 2001, ICQ had over 100 million accounts registered.[2] In April 2010, AOL sold ICQ to Digital Sky Technologies for dollar187.5 million.[3]

% ----------------------------------------------------------------------------
\subsection{Architecture}
% ----------------------------------------------------------------------------
\section{Silc}
% ----------------------------------------------------------------------------
\subsection{History}
% ----------------------------------------------------------------------------
\subsection{Architecture}
% ----------------------------------------------------------------------------
\subsection{Security}

SILC Project develops the Secure Internet Live Conferencing protocol (SILC),
\url{http://silcnet.org/general/}
\url{http://silcnet.org/support/documentation/specs/}

central
\url{http://silcnet.org/support/documentation/wp/silc_protocol.php}


% ----------------------------------------------------------------------------
\section{XMPP/Jabber}
Extensible Messaging and Presence Protocol (XMPP) is an open-standard communications protocol for message-oriented middleware based on XML (Extensible Markup Language).[1] The protocol was originally named Jabber,


The IETF XMPP working group has produced a number of RFC protocol documents: RFC 3920, RFC 3921, RFC 3922, RFC 3923, RFC 4622, RFC 4854, RFC 4979

% ----------------------------------------------------------------------------
\subsection{History}
Jeremie Miller began working on the Jabber technology in 1998 and released the first version of the jabberd server on January 4, 1999.[5] The early Jabber community focused on open-source software, mainly the jabberd server (e.g., version 1.0 in May 2000, version 1.2 in October 2000, and version 1.4 in February 2001), but its major outcome proved the development of the XMPP protocol.

The XMPP WG produced four specifications (RFC 3920, RFC 3921, RFC 3922, RFC 3923), which were approved by the Internet Engineering Steering Group as Proposed Standards in 2004. The XMPP Standards Foundation (formerly the Jabber Software Foundation) is active in developing open XMPP extensions. In 2011, RFC 3920 and RFC 3921 have been superseded by RFC 6120 and RFC 6121 respectively, and RFC 6122 comes to specify the XMPP address format.

by 2003, was used by over ten million people worldwide, according to the XMPP Standards Foundation.[3]

In February 2010, the social-networking site Facebook opened up its chat feature to third-party applications via XMPP.

% ----------------------------------------------------------------------------
\subsection{Architecture}

Decentralization
The architecture of the XMPP network is similar to email; anyone can run their own XMPP server and there is no central master server.

The XMPP network uses a client–server architecture (clients do not talk directly to one another). 

username@example.com.
% ----------------------------------------------------------------------------
\subsection{Security}
TLS for channel encryption and SASL for authentication

% ----------------------------------------------------------------------------
\section{Skype}
More VoIP
% ----------------------------------------------------------------------------
\section{MSN}
More VoIP
% ----------------------------------------------------------------------------
\section{ICQ}
More VoIP


% ----------------------------------------------------------------------------
\section{Security features and Comparision}

\begin{longtable}{|c|c|c|}
\caption{Chat system comparision with security features}\\
\hline
\textbf{Name} & \textbf{IRC} & \textbf{SILC}\\
\hline
\textbf{Single point of attack} & yes & yes\\
\hline
\textbf{Encrypted traffic} & optional & yes\\
\hline
\end{longtable}

All the solutions with objectives.
