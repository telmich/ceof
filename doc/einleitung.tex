%-------------------------------------------------------------------------------
\section{Einleitung}
Das vorliegende Dokument ist im Rahmen der Bachelorarbeit
an der Hochschule für Technik Zürich (HSZ-T) im Jahr 2011 entstanden.
%-------------------------------------------------------------------------------
\subsection{EBS-Eingabe}
Die folgenden Kapitel enthalten die formellen Bestimmungen der Bachelorarbeit
aus dem Einschreibe- und Bewertungssystem der HSZ-T (EBS).
%-------------------------------------------------------------------------------
\subsubsection{Projektname}
Entwicklung einer dezentralen, verschlüsselten Chat-Software.
%-------------------------------------------------------------------------------
\subsubsection{Zusammenfassung}
Da der Typ Unix-Informatiker-KMU nicht nur einen KMU, aber auch einen IT-KMU
Spezialfall darstellt, soll für die Verwaltung dieser Firma eine Individualsoftware
erstellt werden. 
%-------------------------------------------------------------------------------
\subsubsection{Ausgangslage}
Eine Gruppe von Informatikern möchte eine KMU gründen, die sich
auf die Automatisierung von IT-Lösungen im Unix-Bereich spezialisiert.

Damit sich die Firma direkt nach der Gründung auf das Kerngeschäft
konzentrieren kann und die Informatiker Schwächen im Umgang mit
administrative Arbeiten aufweisen, soll ein Konzept-Prototyp
von einer Software erstellt werden, die die Handhabung der
administrativen Prozesse vereinfacht und automatisiert.
%-------------------------------------------------------------------------------
\subsubsection{Ziel der Arbeit}
Es soll eine Softwarelösung entwickelt werden, die die administrativen
Prozesse abbildet und weitestgehend automatisiert instrumentalisiert.
Die entwickelte Software soll primär informatikerfreundlich,
sekundär endnutzernutzbar sein und Schnittstellen zur Erweiterung
durch andere Programme vorsehen.

Die folgenden administrativen Bereiche sollen mindestens abgebildet werden:
\begin{itemize}
\item Projektverwaltung
\item Offertenerstellung
\item Rechnungsgenerierung
\item Zeiterfassung
\end{itemize}
Technische Anforderungen sind des weiteren:
\begin{itemize}
\item Betriebsystemsunabhängiger Zugriff
\item Serverseitig lauffähig unter unixartigem Betriebssystem
\end{itemize}
%-------------------------------------------------------------------------------
\subsubsection{Aufgabenstellung}
\label{aufgabenstellung}
\begin{itemize}
\item Detaillierte Beschreibung der administrativen Bereiche [Requirements Analysis/User] (10\%)
\item Evaluation und Auswahl Entwicklungsumgebung [Requirements Analysis/Technical, Architectual Design] (10\%)
\item Implementierung und Test der Software (55\%)
\item Dokumentation der Software (25\%)
\end{itemize}
%-------------------------------------------------------------------------------
\subsubsection{Erwartetes Resultat}
\begin{itemize}
\item Dokumentation der administrativen Bereiche sowie deren gewünschte Abbildung in der Software
\item Begründung der Auswahl der Entwicklungsumgebung
\item Quelltext der Software 
\item Benutzerhandbuch
\end{itemize}
%-------------------------------------------------------------------------------
\subsection{Geplante Termine} 
\begin{longtable}{|c|c|}
\caption{Geplante Termine}\\
\hline
Thema & Termin\\
\hline
Beginn vom Projekt mit Vorbesprechung ("`Kickoff"') & 2010-10-27 \\
\hline
Zwischenbesprechung ("`Design-Review"') & 2011-01-14 \\
\hline
Abgabe der Arbeit & 2011-04-20 \\
\hline
Projektabschluß mit Vorstellung & 2011-05-04 \\
\hline
\end{longtable}
%-------------------------------------------------------------------------------
\subsection{Formatierungen}
In diesem Dokument werden folgende Formatierung benutzt:
\begin{itemize}
\item \verb=Befehl=
\item \textbf{Betonung einer wichtigen Phrase}
\end{itemize}
\begin{verbatim}
# Kommentar bei der Ausführung
% Eingabe eines Befehls
Ausgabe eines Befehls

Weiterführung einer Zeile \
durch Rückwärtsschrägstrich (Backslash) am Zeilenende
\end{verbatim}
%-------------------------------------------------------------------------------
\subsection{Lizenz}
Dieses Dokument ist lizenziert unter der
\textit{Creative Commons Attribution-ShareAlike 3.0 Unported License}\cite{ccasa3}.
% ok
%-------------------------------------------------------------------------------
\subsection{Geplanter Ablauf}
Die Semesterarbeit ist geplant in folgenden Schritten abzulaufen:
\begin{itemize}
\item Methodische Definition der Anforderungen der Software
\item Definition der administrativen Bereiche
\item Vergleich Eigenherstellung mit Einkauf
\item Auswahl Entwicklungsumgebung zur Umsetzung
\item Umsetzung und Beschreibung der Umsetzung
\item Fazit mit Rück- und Ausblick
\end{itemize}
% Nico: ok
%-------------------------------------------------------------------------------
