% ----------------------------------------------------------------------------
\chapter{Chat Protocol Definition}
    4. Protocol definition paper (containing chat features,
            data types, transport methods, security measures)

\subsection{System Design}
The theory behind EOF.
Search for documents proving/supporting the idea.

Different programs handle different objectives
\subsubsection{Use objectives and derive}
\subsubsection{Objective 1 ...}
\subsubsection{Additional constraints}
Cross-OS. Posix and/or ansi-c. ui changable.
\subsubsection{Key Exchange}
Not implemented, manually.
% ----------------------------------------------------------------------------
\subsection{Intra Machine Intra Program Protocol}
\subsubsection{Interfaces}
Sockets, Environment, Paths, etc.
\subsubsection{Data Types}
\subsubsection{Sub Programs}
Consists of the following sub-programs:
encryption, dictionary, key/peer exchange daemon (if possible in this work),
% ----------------------------------------------------------------------------
\subsection{Sub Programs}
Describe what it does, how it does and where it is implemented.
\subsubsection{Noise / Dictionary / Database}
Generate input for times when there is no user input.
Random or db or or or.
Networt traffic
\subsubsection{Encryption}
Splitting of encryption into a seperate program can make use of
multiple computing ressources.
\subsubsection{User Interface}
Own program, indepentend of core.
\subsubsection{Transports}
Receive/Send

% ----------------------------------------------------------------------------
\subsection{Inter Machine Protocol}
On the "`wire"'. Different transports. Constant transport.
Define name (postcard?!) here. Includes transport specific
header / meta information.

\subsubsection{Variable Transports}
Different transports for one peer.
\subsubsection{List of Supported Transports}
To ensure interoperability, clients which support a specific
protocol version must support all listed transport protocols.
\begin{longtable}{|c|c|c|}
\caption{Transport protocols}\\
\hline
\textbf{Protocol} & \textbf{Description} & \textbf{Supported versions}\\
\hline
tcp & Transmission Control Protocol & 0.1 - 0.1\\
\hline
\end{longtable}

\subsubsection{Variable Peers}
Different routes for every packet
\subsubsection{Constant sending}

\subsubsection{Source based routing}
Either here or in Intra Machine Intra Client
\subsubsection{Peer selection}
Which peers, how many. Constant? May give upper bounds of latency.

8 * 0.5seconds = 4 seconds delay.

\subsubsection{Bandwidth usage}
\begin{verbatim}
>>> messages_size=4*1024
>>> messages_per_second=4
>>> bytes_per_second=messages_per_second*messages_size
>>> print(bytes_per_second)
16384
>>> bytes_per_day=86400*bytes_per_second
>>> print(bytes_per_day)
1415577600
>>> print(bytes_per_day/1024**2)
1350.0
>>> bytes_per_month=month_length*bytes_per_day
>>> print(bytes_per_month/1024**2)
41062.5
>>> print(bytes_per_month/1024**3)
40.10009765625
\end{verbatim}
16 KiB/s or 128 KBit/s, 2 ISDN lines. Around 1.4 GiB per day or
circa 40 GiB per month.

% ----------------------------------------------------------------------------
\subsection{Inter Machine Inter Program Protocol}
After decoding the received packet. Forward, etc.
Based on part os EOF simple data types.
\subsubsection{Message types}
List of messages here
\subsubsection{Message 1}
description here

% ----------------------------------------------------------------------------

\section{Basic data types ("`EOFbdt"')}
This section specifies the basic datatypes used in EOF.
% Nico: 1.0
% -----------------------------------------------------------------------------
\subsection{The zero byte}
The zero byte is a byte with the value 0.
% Nico: 1.0
% -----------------------------------------------------------------------------
\subsection{Line feed}
The line feed, "`\textbackslash{}n"', was used to terminate data
sections, but is \emph{DEPRECATED} now.
% Nico: 1.0
% -----------------------------------------------------------------------------
\subsection{ASCII numbers}
ASCII numbers use the decimal string representation of a number (versus
binary representation, which is \emph{never} used between EOFi and EOFs).
ASCII numbers are often used in a packet header.
ASCII numbers are used to specify the length of the packet (excluding itself).
% Nico: 1.0
% -----------------------------------------------------------------------------
\subsection{Strings in general}
Strings are transmitted without termination (i.e. no new line, no 0 byte),
but are padded with zero bytes, if shorter than the specified length.
The encoding to be used is always \textbf{UTF-8}\cite{utf8}.
% Nico: 1.0
% -----------------------------------------------------------------------------
\subsection{Fixed length strings}
Fixed length strings contain exactly the specified number of bytes:
A 128-byte fixed length string consists of at most 128 bytes of text,
which is then not zero terminated!
If the text it contains is shorter than the specified length,
it must be padded with zero bytes.
% Nico: 1.0
% -----------------------------------------------------------------------------
\subsection{Variable length strings}
The EOF protocol currently does not specify any variable length strings.
All strings are fixed length (see above).
% Nico: 1.0
% -----------------------------------------------------------------------------
\subsection{Noise}
There are many situations in which an EOFi sends out data to the network,
although you did not write a message: In fact, as EOFi \textbf{always}
sends packets in a fixed interval, it needs to have data to encrypt and send.

Noise can be any type of random data. As the current random number generators
are quite expensive, it is recommend to use a huge dictionary, old
messages, logfiles, public emails, etc. for noise input.
% Nico: 1.0
% -----------------------------------------------------------------------------
\subsection{Unused}
To make life harder for attackers we try to make packets always be more or
less the same size. That results in fields being present in a packet, which
are unsued.

Unused fields should be filled up with noise.
% Nico: 1.0
% #############################################################################
\section{EOF simple data types ("`EOFsdt"')}
The following sections define the datatypes used in EOF related
applications. The recommened name for use in source
code is added in parentheses after the human understandable name.
% Nico: 1.0
% -----------------------------------------------------------------------------
\subsection{EOF commands and command fields (mapping table)}
An EOF command is exactly \emph{EOF\_L\_CMD} bytes long (fixed length string)
and contains an ASCII number.

EOF commands are the main method of communication between EOFs and EOFi.

The command field 0 indicates the direction.
The command field 1 indicates the EOF subsystem.
\begin{longtable}{|c|c|c|}
\caption{Command fields}\\
\hline
\textbf{Value} & \textbf{Subsystem} / \textbf{Description} & \textbf{Ref}\\
\hline
1*** & Message is coming from the EOF implementation &\\
\hline
10** & \textbf{eofi2tp}: Transport protocols & p\pageref{eofi2tp}\\
\hline
11** & \textbf{eofi2ui}: User interface & p\pageref{eofi2ui}\\
\hline
12** & \textbf{eofi2crypto}: Crypto engine & p\pageref{eofi2crypto}\\
\hline
13** & \textbf{eofi2noise}: Noise generator & p\pageref{eofi2noise}\\
\hline
2*** & Message is coming from EOF subsystem (internally) &\\
\hline
20** & \textbf{eofi2tp}: Transport protocols & p\pageref{eofi2tp}\\
\hline
21** & \textbf{eofi2ui}: User interface & p\pageref{eofi2ui}\\
\hline
22** & \textbf{eofi2crypto}: Crypto engine & p\pageref{eofi2crypto}\\
\hline
23** & \textbf{eofi2noise}: Noise generator & p\pageref{eofi2noise}\\
\hline
3*** & Message is coming from outside ("`onion packet"')) &\\
\hline
\end{longtable}
The command fields 2 and 3 are defined by the respective subsystem.
% -----------------------------------------------------------------------------
\subsection{Identification string (id)}
\label{idn}
To identify a packet, each packet contains an identification string,
which is \emph{EOF\_L\_ID} bytes long. It may contain only the
following characters:
\begin{itemize}
\item A-Z (alphabet in upper case)
\item a-z (alphabet in lower case)
\item 0-9 (the digits)
\item ! (exclamation mark)
\item - (minus)
\end{itemize}
The EOFs or EOFi may chose freely any of the \emph{68719476736}
possibilities.\footnote{$(26+26+10+2)^6$, as long as EOF\_L\_ID is 6.}
The characters are limited to those characters to allow easy debugging
and to keep the non-binary command layout.
% -----------------------------------------------------------------------------
\subsection{Size (size)}
\index{eof.h (File)}%
All sizes used in this document are "`symbolic sizes"': The real size
is defined in the attached file "`\emph{eof.h}"'.
Developers are advised to use the symbolic name in their programs.

A size is is always represented as an ASCII number found in a
fixed length string of \emph{EOF\_L\_SIZE} bytes.
% Nico: 1.0
% -----------------------------------------------------------------------------
\subsection{Nick name (nick)}
The peer name is a \emph{EOF\_L\_NICKNAME} byte fixed length string.
It is only used internally to give a peer a rememberable name ("`a nick'").
It is never transmitted over the network.
% Nico: 1.0
% -----------------------------------------------------------------------------
\subsection{Group name (group)}
The group name is a \emph{EOF\_L\_GROUP} byte fixed length string.
% Nico: 1.0
% -----------------------------------------------------------------------------
\subsection{Message text (msgtext)}
The message text is a \emph{EOF\_L\_MESSAGE} byte fixed length string.
% Nico: 1.0
% -----------------------------------------------------------------------------
\subsection{Peer address (addr)}
The address of a peer, which is is a \emph{EOF\_L\_ADDRESS}
byte fixed length string. Peer addresses are specified as
URLs as defined in RFC3986\cite{uri-1}. For more information have
a look at section \ref{tp} on page \pageref{tp}.
% Nico: 1.0
% -----------------------------------------------------------------------------
\subsection{Keyid, the fingerprint (keyid)}
A (PGP) fingerprint\footnote{See RFC 2440, 11.2. Key IDs and Fingerprints}
is a \emph{EOF\_L\_KEYID} byte fixed length string.
It does not contain any spaces.
It can be retrieved by issuing the following gpg-command:
\begin{verbatim}
LC_ALL=C gpg --fingerprint  | \
   grep "Key fingerprint =" | \
   sed -e 's/.*=//' -e 's/ //g' 
\end{verbatim}
% Nico: 1.0
% #############################################################################
\section{EOF packets ("`EOFpkg"')}
\label{eofpkg}
% -----------------------------------------------------------------------------
\subsection{Introduction}
No packet (including everything) may exceed the size of \emph{EOF\_L\_PKG\_MAX}.

EOF knows about
\begin{itemize}
\item commands: internal plaintext packets.
\item onions: multiple times encrypted packets including routing information
\item postcards: packets containing transport protocol dependent header
\end{itemize}
Commands are the innermost packet type and only seen within EOFi and EOFs.
Commands are then bundled into a multi layer onion. Each layer contains
commands after decryption.
Onions are put onto a postcard afterwards and are sent out on the network.
\textbf{Only encrypted packets are sent out on the network.}
% -----------------------------------------------------------------------------
\subsection{Commands}
Command packets are used for the communication inside of EOFi and EOFs and
are described in detail in the following sections:
\begin{itemize}
\item eofi2tp, \ref{eofi2tp}, page \pageref{eofi2tp}
\item eofi2ui, \ref{eofi2ui}, page \pageref{eofi2ui}
\item eofi2crypto, \ref{eofi2crypto}, page \pageref{eofi2crypto}
\end{itemize}
% Nico: 1.0
% -----------------------------------------------------------------------------
\subsection{Onions}
Onions are sent out on the network and are multiple times encrypted.
They are building the base for the EOF protocol.
Onions are described in detail in chapter \ref{onions}, page \pageref{onions}.
% -----------------------------------------------------------------------------
\subsection{Postcards}
Onions are afterwards encapsulated into the transport protocol specific
packet type and sent out onto the network.
Postcards are described in detail in chapter \ref{postcards}, page \pageref{postcards}.
