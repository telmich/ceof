\documentclass[12pt,a4paper]{book}
\usepackage[latin1]{inputenc}    % Ascii-Format dieses Dokuments
\usepackage{longtable}           % lange Tabellen
\usepackage{makeidx}             % indexing
\makeindex

\begin{document}
\title{EOF\\Eris Onion Forwarding\\
The secure, peer-to-peer, decentralised
anonymous chat network\thanks{Version 0.6}}
%\date{...}
\author{Nico -telmich- Schottelius}

\maketitle
\tableofcontents
%\newpage

% -----------------------------------------------------------------------------
\chapter{Introduction}
% ----------------------------------------------------------------------------
\section{Copying}
Copy it as you like - send corrections to me.
% Nico: 1.0
% FIXME: Free, see creative commons!
% -----------------------------------------------------------------------------
% -----------------------------------------------------------------------------
\section{Motivation}
% -----------------------------------------------------------------------------
\subsection{Current implementations are not secure}
There are already many different chat protocols available like
\begin{itemize}
\item talk
\item IRC
\item SILC
\item Jabber
\item ICQ
\item Skype
\end{itemize}
Only two of those protocols contains mandority encryption (SILC, Skype), which
still lacks many features for secure chatting.\footnote{Most of them can be
enhanced to use TLS/SSL, but this is not enforced.} Skype and SILC still
depend on a central server architecture. The Skype architecture is not
publicly documentated, the executing binary is encrypted, and the system
depends on a single company, which excluded it from being a secure chat
system.

Because SILC depends on a central server architecture, it was also excluded.
% Nico: 1.0
% FIXME: add references for protocols!
% -----------------------------------------------------------------------------
\subsection{Create more crypto traffic}
We also want to convey usage of PGP: currently PGP is quite seldom.
Thus if you use PGP, you may be conspicuous. We try to make PGP encrypted
packets part of the regular internet traffic, like webtraffic is today.
% Nico: 1.0
