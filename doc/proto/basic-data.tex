\section{Basic data types ("`EOFbdt"')}
This section specifies the basic datatypes used in EOF.
% Nico: 1.0
% -----------------------------------------------------------------------------
\subsection{The zero byte}
The zero byte is a byte with the value 0.
% Nico: 1.0
% -----------------------------------------------------------------------------
\subsection{Line feed}
The line feed, "`\textbackslash{}n"', was used to terminate data
sections, but is \emph{DEPRECATED} now.
% Nico: 1.0
% -----------------------------------------------------------------------------
\subsection{ASCII numbers}
ASCII numbers use the decimal string representation of a number (versus
binary representation, which is \emph{never} used between EOFi and EOFs).
ASCII numbers are often used in a packet header.
ASCII numbers are used to specify the length of the packet (excluding itself).
% Nico: 1.0
% -----------------------------------------------------------------------------
\subsection{Strings in general}
Strings are transmitted without termination (i.e. no new line, no 0 byte),
but are padded with zero bytes, if shorter than the specified length.
The encoding to be used is always \textbf{UTF-8}\cite{utf8}.
% Nico: 1.0
% -----------------------------------------------------------------------------
\subsection{Fixed length strings}
Fixed length strings contain exactly the specified number of bytes:
A 128-byte fixed length string consists of at most 128 bytes of text,
which is then not zero terminated!
If the text it contains is shorter than the specified length,
it must be padded with zero bytes.
% Nico: 1.0
% -----------------------------------------------------------------------------
\subsection{Variable length strings}
The EOF protocol currently does not specify any variable length strings.
All strings are fixed length (see above).
% Nico: 1.0
% -----------------------------------------------------------------------------
\subsection{Noise}
There are many situations in which an EOFi sends out data to the network,
although you did not write a message: In fact, as EOFi \textbf{always}
sends packets in a fixed interval, it needs to have data to encrypt and send.

Noise can be any type of random data. As the current random number generators
are quite expensive, it is recommend to use a huge dictionary, old
messages, logfiles, public emails, etc. for noise input.
% Nico: 1.0
% -----------------------------------------------------------------------------
\subsection{Unused}
To make life harder for attackers we try to make packets always be more or
less the same size. That results in fields being present in a packet, which
are unsued.

Unused fields should be filled up with noise.

