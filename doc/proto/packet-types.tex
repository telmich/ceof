\section{Packets Types ("`EOFpkg"')}
\label{eofpkg}
% -----------------------------------------------------------------------------
\subsection{Introduction}

EOF knows about
\begin{itemize}
\item commands: internal plaintext packets.
\item onions: multiple times encrypted packets including routing information
\item postcards: packets containing transport protocol dependent header
\end{itemize}
Commands are the innermost packet type and only seen within EOFi and EOFs.
Commands are then bundled into a multi layer onion. Each layer contains
commands after decryption.
Onions are put onto a postcard afterwards and are sent out on the network.
\textbf{Only encrypted packets are sent out on the network.}
% -----------------------------------------------------------------------------
\subsection{Commands}
Command packets are used for the communication inside of EOFi and EOFs and
are described in detail in the following sections:
\begin{itemize}
\item eofi2tp, \ref{eofi2tp}, page \pageref{eofi2tp}
\item eofi2ui, \ref{eofi2ui}, page \pageref{eofi2ui}
\item eofi2crypto, \ref{eofi2crypto}, page \pageref{eofi2crypto}
\end{itemize}
% Nico: 1.0
% -----------------------------------------------------------------------------
\subsection{Onions}
Onions are sent out on the network and are multiple times encrypted.
They are building the base for the EOF protocol.
Onions are described in detail in chapter \ref{onions}, page \pageref{onions}.

An onion packet is a (multiple times) encrypted packet.
An onion packet contains at least one plaintext packet, but can also contain
already encrypted packets. It may look like as follows:

All data that is transferred over the network must be encrypted.
% -----------------------------------------------------------------------------
\subsection{Network packets ("`postcards"')}
\label{postcards}
Onions are afterwards encapsulated into the transport protocol specific
packet type and sent out onto the network.
Postcards are described in detail in chapter \ref{postcards}, page \pageref{postcards}.

A postcard "`packet"' contains one onion packet plus the transport protocol
shell.  Postcard packets are the only packet type that is seen by a possible
attacker.  The name postcard was choosen to reflect the fact, that anyone
passing the postcard can read what is written on it.

All packets must be signed by the sender and encrypted for the receiver.
The different datatypes are just concatenated in the order described.
The following description of the content describes the pakckets
in their unencrypted form.
% Nico: 1.0

on-the-wire


The EOF messages described in the previous section are multiple times
encrypted and assembled according to the calculated source route.

These packets are codenamed "`postcards"', as it is assumed they can be read
by an attacker.
