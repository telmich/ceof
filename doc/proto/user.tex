\section{The user interface ("`ui2user"')}
This section specifies the recommended appereance of a 
user interface to the user. Due to the modular architecture it is likely
that different user interfaces will appear and this section defines
commands that \textbf{should} be implemented by every user interface
to provide a consistent look and feel towards the user.
% Nico: 1.0
%---------------------------------------------------------------------
\subsection{Minimal philosophy}
All EOF compliant user interfaces should support the named commands, so the
user can change the UI, but can be sure that this minimal amount of
commands is always available.

Every user interface \emph{may} add additional input methods or commands.
% Nico: 1.0
%---------------------------------------------------------------------
\subsection{Command line / Text input}
Every user interface should provide a line to enter commands.
%---------------------------------------------------------------------
\subsection{Commands in general}
All commands begin with a "`/"' as first character (adopted from IRC).
% Nico: 1.0
%---------------------------------------------------------------------
\subsection{Command length}
The user interfaces need only to accept commands up to a length of
\emph{EOF\_L\_UI\_INPUT}.
If the user inputs longer commands, they should be truncated
to \emph{EOF\_L\_UI\_INPUT} bytes.
% Nico: 1.0
%---------------------------------------------------------------------
\subsection{Send text}
If the entered text does not begin with a "`/"', it should be treated
as a message to the current selected destination (either
a \emph{peer} or \emph{group} of peers.
% Nico: 1.0
%---------------------------------------------------------------------
\subsection{/help}
The help command should print a short usage information to the user.
\index{Command!/help}%
\begin{longtable}{|c|c|c|c|}
\caption{/help parameters}\\
\hline
\textbf{Parameter} & \textbf{Type} & \textbf{Description} & \textbf{Example}\\
\hline
Command & EOFsdt & Optional command named without starting / & /help connect\\
\hline
\end{longtable}
FIXME: specify optional type
%---------------------------------------------------------------------
\subsection{/connect $<$host$>$ $<$port$>$}
The connect command can be used to connect to an EOF
implementation.
\index{Command!/connect}%
\begin{longtable}{|c|c|c|c|}
\caption{/connect parameters}\\
\hline
\textbf{Parameter} & \textbf{Type} & \textbf{Description} & \textbf{Example}\\
\hline
Host & EOFsdt & Host to connect to & 127.0.0.1\\
\hline
Port & EOFsdt & Port to connect to & 4242\\
\hline
\end{longtable}

%---------------------------------------------------------------------
\subsection{/disconnect}
The disconnect command is used to disconnect from the chat server.
Parameter: None.
%---------------------------------------------------------------------
\subsection{/peer add $<$nick$>$ $<$initial addr$>$ $<$keyid$>$}
Add the peer as \textit{name} to the list of known peers.

\index{Command!/peer add}%
\begin{longtable}{|c|c|c|c|}
\caption{/peer add parameters}\\
\hline
\textbf{Parameter} & \textbf{Type} & \textbf{Description} & \textbf{Example}\\
\hline
Nick & EOFsdt & Name you identify the peer with & telmich\\
\hline
Initial addr & EOFsdt & Where we can make the first contact & tcp:10.0.42.42:4242\\
\hline
Keyid & EOFsdt & The PGP fingerprint of the peers public key & F27987E34E66...\\
\hline
\end{longtable}

\subsubsection{Example}
\begin{verbatim}
/peer add telmich tcp:10.0.42.42:4242 F27987E34E7866B2BA39C2FD793EB8FC325251FE
\end{verbatim}
% Nico: 1.0
%---------------------------------------------------------------------
\subsection{/peer send $<$nick$>$ $<$msgtext...$>$}
Send message \textit{msgtext} to peer \textit{nick}.

\index{Command!/peer send}%
\begin{longtable}{|c|c|c|c|}
\caption{/peer send parameters}\\
\hline
\textbf{Parameter} & \textbf{Type} & \textbf{Description} & \textbf{Example}\\
\hline
Nick & EOFsdt & Name you identify the peer with & telmich\\
\hline
Msgtext & EOFsdt & The message itself & Hallo, wie geht es Dir?\\
\hline
\end{longtable}

\subsubsection{Example}
\begin{verbatim}
/peer send telmich Hallo, wie geht es Dir?
\end{verbatim}
% Nico: 1.0
%---------------------------------------------------------------------
\subsection{/peer rename $<$oldnick$>$ $<$newnick$>$}
Renames the peer.
\index{Command!/peer rename}%
\begin{longtable}{|c|c|c|c|}
\caption{/peer rename parameters}\\
\hline
\textbf{Parameter} & \textbf{Type} & \textbf{Description} & \textbf{Example}\\
\hline
Oldnick & EOFsdt & Old nick name & susi\\
\hline
Newnick & EOFsdt & New nick name & heinz\\
\hline
\end{longtable}

\subsubsection{Example}
\begin{verbatim}
/peer rename susi heinz
\end{verbatim}
% Nico: 1.0
%---------------------------------------------------------------------
\subsection{/peer show $<$nick$>$}
Display detailled information about peer.
\index{Command!/peer show}%
\begin{longtable}{|c|c|c|c|}
\caption{/peer rename parameters}\\
\hline
\textbf{Parameter} & \textbf{Type} & \textbf{Description} & \textbf{Example}\\
\hline
Nick name & EOFsdt & Nick name as known by EOFi & karl-otto\\
\hline
\end{longtable}

\subsubsection{Example}
\begin{verbatim}
/peer show karl-otto
\end{verbatim}
% Nico: 1.0
%---------------------------------------------------------------------
\subsection{/peer list}
List currently known peers. This command does not accept any parameters.

\subsubsection{Example}
\begin{verbatim}
/peer list
\end{verbatim}
% Nico: 1.0
%---------------------------------------------------------------------
\subsection{/exit}
Request the user interface to exit. It will deregister from EOFi,
but EOFi will continue to run (even with no user interface attached).
\subsubsection{Example}
\begin{verbatim}
/exit
\end{verbatim}
% Nico: 1.0
%---------------------------------------------------------------------
\subsection{/quit}
The UI tells EOFi to quit.
The EOFi will tell all EOFs to quit and quits.
Afterwards the UI will also quit.
\subsubsection{Example}
\begin{verbatim}
/quit
\end{verbatim}
% Nico: 1.0
%---------------------------------------------------------------------
\subsection{Aliases}
Aliases may optionally be provided by the UI. If an UI provides
support for aliases, it must implement the "`\emph{/alias}"' command.

The following aliases should be provided by default,
to aid new users using EOF.
% Nico: 1.0
%---------------------------------------------------------------------
\subsection{/alias $<aliasname>$ $<command...>$}
This command should be used to setup aliases.
% Nico: 1.0
%---------------------------------------------------------------------
\subsection{/msg $<nick>$ $<msgtext>$}
Should be an alias for \textit{/peer send $<$nick$>$ $<$msgtext$>$}
% Nico: 1.0
%---------------------------------------------------------------------
\subsection{/whois $<nick>$}
Should be an alias for \textit{/peer show $<$nick$>$}
% Nico: 1.0


