\section{The User Interface ("`ui2user"')}
This section specifies the appereance of the user interface to the user.
% Nico: 1.0
%---------------------------------------------------------------------
\subsection{Interface}
The UI is running as a ncurses application and prompts for input
on a specific line.
All commands start with a "`/"'.
%---------------------------------------------------------------------
\subsection{UI Command: /help}
The /help command prints a short usage information to the user.

\subsubsection{Example}
\begin{verbatim}
\end{verbatim}
% FIXME: Example

%---------------------------------------------------------------------
\subsection{UI Command: /connect $[$host$]$ $[$port$]$}
The connect command can be used to connect to the chat server.
Host and port are optional. If omitted, the saved host and/or
port will be used. This command uses message 2100.
\subsubsection{Example}
\begin{verbatim}
/connect 127.0.0.1 4242
\end{verbatim}
% Nico: Code & Doc
%---------------------------------------------------------------------
\subsection{UI Command: /quit}
Request the user interface to exit. It will deregister from the CS.
This command uses message 2101.
\subsubsection{Example}
\begin{verbatim}
/quit
\end{verbatim}
% Nico: 1.0
%---------------------------------------------------------------------
\subsection{UI Command: /allquit}
The UI tells the CS and all connected UIs (including itself) to quit.
This command uses message 2199.
\subsubsection{Example}
\begin{verbatim}
/allquit
\end{verbatim}
% Nico: 1.0
%---------------------------------------------------------------------
\subsection{UI Command: /peer add $<$name$>$ $<$address$>$ $<$keyid$>$}
Add the peer with the given name \textit{name} to the list of known peers.

\index{UI Command!/peer add}%
\begin{longtable}{|c|c|c|c|}
\caption{UI Command: /peer add parameters}\\
\hline
\textbf{Parameter} & \textbf{Type} & \textbf{Description} & \textbf{Example}\\
\hline
Peer name & EOFsdt: name & Name you identify the peer with & telmich\\
\hline
Address & EOFsdt: address & Where we can make the first contact & tcp://10.0.42.42:4242\\
\hline
Keyid & EOFsdt: keyid & The PGP fingerprint of the peers public key & F27987E34E66...\\
\hline
\end{longtable}

\subsubsection{Example}
\begin{verbatim}
/peer add telmich tcp//:10.0.42.42:4242 F27987E34E7866B2BA39C2FD793EB8FC325251FE
\end{verbatim}
% Nico: 1.0
%---------------------------------------------------------------------
\subsection{UI Command: /peer send $<$name$>$ $<$msgtext$>$}
Send message \textit{msgtext} to peer \textit{nick}.

\index{UI Command!/peer send}%
\begin{longtable}{|c|c|c|c|}
\caption{UI Command: /peer send parameters}\\
\hline
\textbf{Parameter} & \textbf{Type} & \textbf{Description} & \textbf{Example}\\
\hline
Name & EOFsdt: name & Name you identify the peer with & telmich\\
\hline
Msgtext & EOFsdt: msgtxt & The message itself & Hallo, wie geht es Dir?\\
\hline
\end{longtable}

\subsubsection{Example}
\begin{verbatim}
/peer send telmich Hallo, wie geht es Dir?
\end{verbatim}
% Nico: 1.0
%---------------------------------------------------------------------
\subsection{UI Command: /peer rename $<$oldnick$>$ $<$newnick$>$}
Renames the peer.
\index{UI Command!/peer rename}%
\begin{longtable}{|c|c|c|c|}
\caption{UI Command: /peer rename parameters}\\
\hline
\textbf{Parameter} & \textbf{Type} & \textbf{Description} & \textbf{Example}\\
\hline
Oldnick & EOFsdt & Old nick name & susi\\
\hline
Newnick & EOFsdt & New nick name & heinz\\
\hline
\end{longtable}

\subsubsection{Example}
\begin{verbatim}
/peer rename susi heinz
\end{verbatim}
% Nico: 1.0
%---------------------------------------------------------------------
\subsection{UI Command: /peer show $<$nick$>$}
Display detailled information about peer.
\index{UI Command!/peer show}%
\begin{longtable}{|c|c|c|c|}
\caption{UI Command: /peer rename parameters}\\
\hline
\textbf{Parameter} & \textbf{Type} & \textbf{Description} & \textbf{Example}\\
\hline
Nick name & EOFsdt & Nick name as known by EOFi & karl-otto\\
\hline
\end{longtable}

\subsubsection{Example}
\begin{verbatim}
/peer show karl-otto
\end{verbatim}
% Nico: 1.0
%---------------------------------------------------------------------
\subsection{UI Command: /peer list}
List currently known peers. This command does not accept any parameters.

\subsubsection{Example}
\begin{verbatim}
/peer list
\end{verbatim}
% Nico: 1.0
% %---------------------------------------------------------------------
% \subsection{Aliases}
% Aliases may optionally be provided by the UI. If an UI provides
% support for aliases, it must implement the "`\emph{/alias}"' command.
% 
% The following aliases should be provided by default,
% to aid new users using EOF.
% % Nico: 1.0
% %---------------------------------------------------------------------
% \subsection{/alias $<aliasname>$ $<command...>$}
% This command should be used to setup aliases.
% % Nico: 1.0
% %---------------------------------------------------------------------
% \subsection{/msg $<nick>$ $<msgtext>$}
% Should be an alias for \textit{/peer send $<$nick$>$ $<$msgtext$>$}
% % Nico: 1.0
% %---------------------------------------------------------------------
% \subsection{/whois $<nick>$}
% Should be an alias for \textit{/peer show $<$nick$>$}
% % Nico: 1.0
