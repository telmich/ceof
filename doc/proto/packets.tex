% -----------------------------------------------------------------------------
\section{Network packets ("`postcards"')}
\label{postcards}
All data that is transferred over the network must be encrypted.
The EOF messages described in the previous section are multiple times
encrypted and assembled according to the calculated source route.
These packets are codenamed "`postcards"', as it is assumed they can be read
by an attacker.
% Nico: 1.0
% -----------------------------------------------------------------------------
\subsection{Onion packets}
An onion packet is a (multiple times) encrypted packet.
An onion packet contains at least one plaintext packet, but can also contain
already encrypted packets. It may look like as follows:
% -----------------------------------------------------------------------------
\subsubsection{Example onion packet}
\begin{itemize}
\item 
\end{itemize}
% Nico: FIXME for 1.0
% -----------------------------------------------------------------------------
\subsection{Postcard packets}
A postcard "`packet"' contains one onion packet plus the transport protocol
shell.  Postcard packets are the only packet type that is seen by a possible
attacker.  The name postcard was choosen to reflect the fact, that anyone
passing the postcard can read what is written on it.

All packets must be signed by the sender and encrypted for the receiver.
The different datatypes are just concatenated in the order described.
The following description of the content describes the pakckets
in their unencrypted form.
% Nico: 1.0

