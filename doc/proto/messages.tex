% -----------------------------------------------------------------------------
\section{EOF messages ("`3***: onions"')}
Onions are the result of decrypting an incoming packet (respective vice versa
when sending out).
% -----------------------------------------------------------------------------
\subsection{Introduction}
The following types are defined:

\begin{itemize}
\item 3000: Drop packet
\item 3001: Forward packet
\item 3002: Message / drop packet
\item 3003: Message / forward packet
\item 3004: Acknowledge
\end{itemize}
% Nico: 1.0
% -----------------------------------------------------------------------------
\subsection{Parameters}
All \textbf{3***} packets have the same length and contain the same fields:
\begin{longtable}{|c|c|c|c|c|}
\caption{Command 3*** parameters}\\
\hline
\textbf{Command} & \textbf{id} & \textbf{addr} & \textbf{group} & \textbf{msgtext}\\
\hline
3000 & - & - & - & - \\
\hline
3001 & - & x & - & - \\
\hline
3002 & x & - & x & x \\
\hline
3003 & x & x & x & x \\
\hline
3004 & x & - & - & -\\
\hline
\end{longtable}

\begin{itemize}
\item -: Not used
\item x: Used
\end{itemize}

\begin{longtable}{|c|c|c|c|}
\caption{Command 3*** parameters}\\
\hline
\textbf{Parameter} & \textbf{Type} & \textbf{Description} & \textbf{Example}\\
\hline
id & EOFsdt & Packet id & alg4f!\\
\hline
addr & EOFsdt & Adress of next peer & tcp:123.123.123.132:8080\\
\hline
group & EOFsdt & The destination group & !eof\\
\hline
msgtext & EOFsdt & The message & Hallo, mein Freund!\\
\hline
\end{longtable}
% Nico: 1.0
% -----------------------------------------------------------------------------
\subsection{3000: Drop packet}
You are the last recipient and there's nothing interesting left.
Just drop the packet and continue work.
% Nico: 1.0
% -----------------------------------------------------------------------------
\subsection{3001: Forward packet}
If a peer receives a packet with the command 3001, it simply forwards
the message to the peer specified in the \textbf{addr} field.
All data contained in the message
is noise. After the message has been forwarded to the next peer, it
should be dropped. If the peer is unreachable, the message should also
be dropped.
% Nico: 1.0
% -----------------------------------------------------------------------------
\subsection{3002: Message / drop packet}
This packet contains a messages to be read and does not need to be forwarded
anymore: You are the last peer in the chain.
\begin{itemize}
\item If the first byte of the group is the zero byte, the message
is a private message (i.e. only sent to you).
\item If the first byte of the group field is \textbf{non-zero} the message
is addressed to the specified group.
\end{itemize}
% Nico: 1.0
% -----------------------------------------------------------------------------
\subsection{3003: Message / forward packet}
The command 3002 is a combination of command 3001 and 3002.
% -----------------------------------------------------------------------------
\subsection{3004: Acknowledge}
Acknowledge the receipt of a received message. The ID must be the same as
the one specified in the original messages packet.
Every message packet must be acknowledged.
% -----------------------------------------------------------------------------
\subsection{Routing}
This version of EOF does not know how to create a route.
All packages are transferred directly to the final peer (which is an
incredible big huge bug) in this version of EOF. Source routing will be
described and defined in future versions.
% Nico: 1.0

