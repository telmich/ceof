% -----------------------------------------------------------------------------
\section{Messages Types ("`EOFmsg"')}
\label{eofmsg}
The following message types are defined:

\begin{itemize}
\item 3000: Drop packet
\item 3001: Forward packet
\item 3002: Message / drop packet
\item 3003: Message / forward packet
\item 3004: Acknowledge
\end{itemize}
% Nico: 1.0
% -----------------------------------------------------------------------------
\subsection{Parameter Overview}
All messages have the same length and contain the same fields.
Though not all fields are being used in every type, as seen in table
\ref{eofmessageparameteroverview}.
\begin{longtable}{|c|c|c|c|c|c|}
\caption{EOF Message Parameter Usage}
\label{eofmessageparameteroverview}
\\
\hline
\textbf{Command} & \textbf{version} & \textbf{id} & \textbf{addr} & \textbf{group} & \textbf{msgtext}\\
\hline
3000 &x & - & - & - & - \\
\hline
3001 &x & - & x & - & - \\
\hline
3002 &x & x & - & x & x \\
\hline
3003 &x & x & x & x & x \\
\hline
3004 &x & x & - & - & -\\
\hline
\end{longtable}
Where 
\begin{itemize}
\item \textbf{-} means \textbf{not used}
\item and \textbf{x} means \textbf{used}.
\end{itemize}
In table \ref{eofmessagedescription} the messages
parameters are described.
\begin{longtable}{|c|c|c|c|}
\caption{EOF Message Parameter Description}
\label{eofmessagedescription}
\\
\hline
\textbf{Parameter} & \textbf{Type} & \textbf{Description} & \textbf{Example}\\
\hline
version & EOFsdt & EOF Version & 0\\
\hline
id & EOFsdt & Packet id & alg4f!\\
\hline
addr & EOFsdt & Address of next peer & tcp:123.123.123.132:8080\\
\hline
group & EOFsdt & The destination group & !eof\\
\hline
msgtext & EOFsdt & The message & Hallo, mein Freund!\\
\hline
\end{longtable}
Version is always 0. Is though for future changes in the protocol.
Version is two bytes.
% Nico: 1.0
% -----------------------------------------------------------------------------
\subsection{3000: Drop packet}
You are the last recipient and there's nothing interesting left.
Just drop the packet and continue work.
% Nico: 1.0
% -----------------------------------------------------------------------------
\subsection{3001: Forward packet}
If a peer receives a packet with the command 3001, it simply forwards
the message to the peer specified in the \textbf{addr} field.
All data contained in the message
is noise. After the message has been forwarded to the next peer, it
should be dropped. If the peer is unreachable, the message should also
be dropped.
% Nico: 1.0
% -----------------------------------------------------------------------------
\subsection{3002: Message / drop packet}
This packet contains a messages to be read and does not need to be forwarded
anymore: You are the last peer in the chain.
%% Add group support back later
%% \begin{itemize}
%% \item If the first byte of the group is the zero byte, the message
%% is a private message (i.e. only sent to you).
%% \item If the first byte of the group field is \textbf{non-zero} the message
%% is addressed to the specified group.
%% \end{itemize}
% Nico: 1.0
% -----------------------------------------------------------------------------
\subsection{3003: Message / forward packet}
The command 3002 is a combination of command 3001 and 3002
and instructs the peer to read the message text and to forward the rest of
the packet.
% -----------------------------------------------------------------------------
\subsection{3004: Acknowledge}
Acknowledge the receipt of a received message. The ID must be the same as
the one specified in the original messages packet.
Every message packet must be acknowledged.
% -----------------------------------------------------------------------------
