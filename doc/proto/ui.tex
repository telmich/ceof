\section{Interface between the chat server and the user interface ("`cs2ui"')}
\label{eofi2ui}
This section specifies the commands used between the user interface (UI)
and chat server (CS).
% Nico: 1.0
% -----------------------------------------------------------------------------
\subsection{Connection}
The chat server provides a TCP listener on port 6667, to which the UI connects 
to. Alternate ports may be used, but need to be specified explicitly.
% -----------------------------------------------------------------------------
\subsection{Commands: Classes and packet description}
All packets exchanged between EOFi and the UI are formatted as
\textit{EOF commands}. EOF commands always begin with 4 bytes, 
which contain a numeric code encoded in UTF-8.

Commands sent by the server begin with \textbf{11}, commands send by the
UI begin with \textbf{21}.
% -----------------------------------------------------------------------------
\subsection{ID}
% -----------------------------------------------------------------------------
\subsection{1100: Acknowledge}
The is a general acknowledge answer. 
The request with the same \emph{ID} as the packet was successful.
\subsubsection{Parameters}
\begin{longtable}{|c|c|c|c|}
\caption{Command 1100 parameters}\\
\hline
\textbf{Parameter} & \textbf{Type} & \textbf{Description} & \textbf{Example}\\
\hline
ID & EOFsdt: id & packet id & afdb12\\
\hline
\end{longtable}
\subsubsection{Example}
\begin{verbatim}
1100abfudh
\end{verbatim}
% Nico: 1.0
% -----------------------------------------------------------------------------
\subsection{1101: Failure}
The is a general failure answer. The request with the
same \emph{ID} as the packet failed.
Details are specified in the reason message.
\subsubsection{Parameters}
\begin{longtable}{|c|c|c|c|}
\caption{Command 1101 parameters}\\
\hline
\textbf{Parameter} & \textbf{Type} & \textbf{Description} & \textbf{Example}\\
\hline
ID & EOFsdt: id & packet id & afdb12\\
\hline
Reason & EOFsdt: msgtxt & Why the connection was refused & Too many UIs connected.\\
\hline
\end{longtable}
If the failed command was "`2100"', EOFi will close the socket afterwards.
\subsubsection{Example}
\begin{verbatim}
1101abfudhSome error\0\0\0\0\0\0\0\0\0\0\0\0\0\0\0\0\0\0\0\0\0\0\0\0\0\0\0
\0\0\0\0\0\0\0\0\0\0\0\0\0\0\0\0\0\0\0\0\0\0\0\0\0\0\0\0\0\0\0\0\0\0\0\0\0
\0\0\0\0\0\0\0\0\0\0\0\0\0\0\0\0\0\0\0\0\0\0\0\0\0\0\0\0\0\0\0\0\0\0\0\0\0
\0\0\0\0\0\0\0\0\0
\end{verbatim}
% Nico: 1.0
% -----------------------------------------------------------------------------
\subsection{1102: Exit requested}
This is a  shutdown request to the UI.
After this message EOFi will exit and there is no communication possible.
\subsubsection{Parameters}
\begin{longtable}{|c|c|c|c|}
\caption{Command 1102 parameters}\\
\hline
\textbf{Parameter} & \textbf{Type} & \textbf{Description} & \textbf{Example}\\
\hline
ID & EOFsdt: id & packet id & afdb12\\
\hline
\end{longtable}
\subsubsection{Example}
\begin{verbatim}
1102abf93a
\end{verbatim}
% Nico: 1.0
% -----------------------------------------------------------------------------
\subsection{1103: Recieved message}
This message is issued by EOFi, if a message is received.
\subsubsection{Parameters}
\index{Command!1103}
\begin{longtable}{|c|c|c|c|}
\caption{Command 1103 parameters}\\
\hline
\textbf{Parameter} & \textbf{Type} & \textbf{Description} & \textbf{Example}\\
\hline
ID & EOFsdt: id & packet id & afdb12\\
\hline
nick & EOFsdt & The sender & telmich\\
\hline
msgtext & EOFsdt & The message & Hallo, mein Freund!\\
\hline
\end{longtable}
\subsubsection{Example}
\begin{verbatim}
1103abfudhtelmich\0\0\0\0\0\0\0\0\0\0\0\0\0\0\0\0\0\0\0\0\0\0\0\0\0\0\0\0\0
\0\0\0\0\0\0\0\0\0\0\0\0\0\0\0\0\0\0\0\0\0\0\0\0\0\0\0\0\0\0\0\0\0\0\0\0\0
\0\0\0\0\0\0\0\0\0\0\0\0\0\0\0\0\0\0\0\0\0\0\0\0\0\0\0\0\0\0\0\0\0\0\0\0\0
\0\0\0\0\0\0\0\0\0\0Hallo!\0\0\0\0\0\0\0\0\0\0\0\0\0\0\0\0\0\0\0\0\0\0\0\0
\0\0\0\0\0\0\0\0\0\0\0\0\0\0\0\0\0\0\0\0\0\0\0\0\0\0\0\0\0\0\0\0\0\0\0\0\0
\0\0\0\0\0\0\0\0\0\0\0\0\0\0\0\0\0\0\0\0\0\0\0\0\0\0\0\0\0\0\0\0\0\0\0\0\0
\0\0\0\0\0\0\0\0\0\0\0\0\0\0\0\0
\end{verbatim}
\subsubsection{Possible answers}
\begin{itemize}
\item None: EOFi does not resend this packet if the UI lost it.
\end{itemize}
% Nico: 1.0
% -----------------------------------------------------------------------------
\subsection{1104: List of peers}
This is the answer to command \emph{2106} and thus contains the
same ID, as the \emph{2106} request command.
\subsubsection{Parameters}
\begin{longtable}{|c|c|c|c|}
\caption{Command 1104 parameters}\\
\hline
\textbf{Parameter} & \textbf{Type} & \textbf{Description} & \textbf{Example}\\
\hline
ID & EOFsdt: id & packet id & afdb12\\
\hline
Number of peers (nop) & EOFsdt: size & How many peers follow & 20\\
\hline
$nop * Peer$ & EOFsdt: nick & The nickname & telmich\\
\hline
\end{longtable}
The last field is repeated as many times as specified in number of peers.
% Nico: 1.0
% -----------------------------------------------------------------------------
\subsection{1105: Peer information}
This is the answer to command \emph{2105} and thus contains the
same ID, as the \emph{2105} request command.
\subsubsection{Parameters}
\begin{longtable}{|c|c|c|c|}
\caption{Command 1105 parameters}\\
\hline
\textbf{Parameter} & \textbf{Type} & \textbf{Description} & \textbf{Example}\\
\hline
ID & EOFsdt: id & packet id & afdb12\\
\hline
Keyid & EOFsdt:keyid & This peers pgp-keyid & 389E5481065EAA253...\\
\hline
Number of addresses (noa) & EOFsdt: size & & 1\\
\hline
$noa * address$ & EOFsdt: addr & Adress of peer & tcp:127.0.0.1:4243\\
\hline
\end{longtable}
The last field is repeated as often, as specified in the number of addresses
field.
% Nico: 1.0
% -----------------------------------------------------------------------------
\subsection{1106: Peer renamed}
This is the answer to command \emph{2104} and thus contains the
same ID, as the \emph{2104} request command. It is sent out to
\textbf{all} connected user interfaces.
\subsubsection{Parameters}
\begin{longtable}{|c|c|c|c|}
\caption{Command 1106 parameters}\\
\hline
\textbf{Parameter} & \textbf{Type} & \textbf{Description} & \textbf{Example}\\
\hline
ID & EOFsdt: id & packet id & afdb12\\
\hline
Oldnick & EOFsdt & Old nick name & susi\\
\hline
Newnick & EOFsdt & New nick name & heinz\\
\hline
\end{longtable}
\subsubsection{Possible answers}
\begin{itemize}
\item None
\end{itemize}
% Nico: 1.0
% -----------------------------------------------------------------------------
\subsection{2100: Register user interface}
This must be the \emph{first} message sent by the UI. If the answer is not
\emph{1100}, the UI should close the socket afterwards.

\subsubsection{Parameters}
\begin{longtable}{|c|c|c|c|}
\caption{Command 2100 parameters}\\
\hline
\textbf{Parameter} & \textbf{Type} & \textbf{Description} & \textbf{Example}\\
\hline
ID & EOFsdt: id & packet id & afdb12\\
\hline
Name & EOFsdt: UINAME & Name of the UI & ceofui\\
\hline
\end{longtable}

\subsubsection{Possible answers}
\begin{itemize}
\item 1100
\item 1101
\end{itemize}
% Nico: 1.0
% -----------------------------------------------------------------------------
\subsection{2101: Deregister user interface}

\subsubsection{Parameters}
\begin{itemize}
\item none
\end{itemize}

\subsubsection{Possible answers}
\begin{itemize}
\item none
\end{itemize}

EOFi will close the connection to the UI after receiving this message.

% Nico: 1.0
% -----------------------------------------------------------------------------
\subsection{2102: /peer add}
The UI adds a peer to the list of known peers.

\subsubsection{Parameters}
\index{Command!2102: /peer add}%
\begin{longtable}{|c|c|c|c|}
\caption{2102: /peer add parameters}\\
\hline
\textbf{Parameter} & \textbf{Type} & \textbf{Description} & \textbf{Example}\\
\hline
ID & EOFsdt: id & packet id & afdb12\\
\hline
Nick & EOFsdt: nick & Name you identify the peer with & telmich\\
\hline
Address & EOFsdt: addr & Where we can make the first contact & tcp:10.0.42.42:4242\\
\hline
Keyid & EOFsdt: keyid & PGP fingerprint of the peers key & F27987E34E66...\\
\hline
\end{longtable}

\subsubsection{Possible answers}
\begin{itemize}
\item 1100
\item 1101
\end{itemize}
% Nico: 1.0
% -----------------------------------------------------------------------------
\subsection{2103: /peer send}
The UI wants to submit a message to a peer.

\subsubsection{Parameters}
\index{Command!/peer send}%
\begin{longtable}{|c|c|c|c|}
\caption{2103: /peer send parameters}\\
\hline
\textbf{Parameter} & \textbf{Type} & \textbf{Description} & \textbf{Example}\\
\hline
ID & EOFsdt: id & packet id & afdb12\\
\hline
Nick & EOFsdt: nick & Name you identify the peer with & telmich\\
\hline
Message & EOFsdt: msgtxt & The message itself & Hallo, wie geht es Dir?\\
\hline
\end{longtable}

\subsubsection{Possible answers}
\begin{itemize}
\item 1100
\item 1101
\end{itemize}
% Nico: 1.0
% -----------------------------------------------------------------------------
\subsection{2104: /peer rename}
The UI wants to rename a peer.

\subsubsection{Parameters}
\index{Command!/peer rename}%
\begin{longtable}{|c|c|c|c|}
\caption{/peer rename parameters}\\
\hline
\textbf{Parameter} & \textbf{Type} & \textbf{Description} & \textbf{Example}\\
\hline
ID & EOFsdt: id & packet id & afdb12\\
\hline
Oldnick & EOFsdt & Old nick name & susi\\
\hline
Newnick & EOFsdt & New nick name & heinz\\
\hline
\end{longtable}

\subsubsection{Possible answers}
\begin{itemize}
\item 1106
\item 1101
\end{itemize}

% -----------------------------------------------------------------------------
\subsection{2105: /peer show}
The UI requests details about a peer.

\subsubsection{Parameters}
\begin{longtable}{|c|c|c|c|}
\caption{2105: /peer show parameters}\\
\hline
\textbf{Parameter} & \textbf{Type} & \textbf{Description} & \textbf{Example}\\
\hline
ID & EOFsdt: id & packet id & afdb12\\
\hline
Nick name & EOFsdt & Nick name, as known by EOFi & karl-otto\\
\hline
\end{longtable}

\subsubsection{Possible answers}
\begin{itemize}
\item 1101
\item 1105
\end{itemize}
% Nico: 1.0
% -----------------------------------------------------------------------------
\subsection{2106: /peer list}
The UI requests the list of known peers.

\subsubsection{Parameters}
\begin{longtable}{|c|c|c|c|}
\caption{2106: /peer list parameters}\\
\hline
\textbf{Parameter} & \textbf{Type} & \textbf{Description} & \textbf{Example}\\
\hline
ID & EOFsdt: id & packet id & afdb12\\
\hline
\end{longtable}

\subsubsection{Possible answers}
\begin{itemize}
\item 1101
\item 1104
\end{itemize}
% Nico: 1.0
% -----------------------------------------------------------------------------
\subsection{2199: /quit}
The user interface requests EOFi and all other EOFs to exit.
The EOFi will not answer, but send an exit request to all other
EOFs.

\subsubsection{Parameters}
\begin{longtable}{|c|c|c|c|}
\caption{2199: /quit parameters}\\
\hline
\textbf{Parameter} & \textbf{Type} & \textbf{Description} & \textbf{Example}\\
\hline
ID & EOFsdt: id & packet id & afdb12\\
\hline
\end{longtable}

\subsubsection{Possible answers}
\begin{itemize}
\item none
\end{itemize}
% Nico: 1.0

