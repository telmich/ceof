% #############################################################################
\section{EOF simple data types ("`EOFsdt"')}
The following sections define the datatypes used in EOF related
applications. The recommened name for use in source
code is added in parentheses after the human understandable name.
% Nico: 1.0
% -----------------------------------------------------------------------------
\subsection{EOF commands and command fields (mapping table)}
An EOF command is exactly \emph{EOF\_L\_CMD} bytes long (fixed length string)
and contains an ASCII number.

EOF commands are the main method of communication between EOFs and EOFi.

The command field 0 indicates the direction.
The command field 1 indicates the EOF subsystem.
\begin{longtable}{|c|c|c|}
\caption{Command fields}\\
\hline
\textbf{Value} & \textbf{Subsystem} / \textbf{Description} & \textbf{Ref}\\
\hline
1*** & Message is coming from the EOF implementation &\\
\hline
10** & \textbf{eofi2tp}: Transport protocols & p\pageref{eofi2tp}\\
\hline
11** & \textbf{eofi2ui}: User interface & p\pageref{eofi2ui}\\
\hline
12** & \textbf{eofi2crypto}: Crypto engine & p\pageref{eofi2crypto}\\
\hline
13** & \textbf{eofi2noise}: Noise generator & p\pageref{eofi2noise}\\
\hline
2*** & Message is coming from EOF subsystem (internally) &\\
\hline
20** & \textbf{eofi2tp}: Transport protocols & p\pageref{eofi2tp}\\
\hline
21** & \textbf{eofi2ui}: User interface & p\pageref{eofi2ui}\\
\hline
22** & \textbf{eofi2crypto}: Crypto engine & p\pageref{eofi2crypto}\\
\hline
23** & \textbf{eofi2noise}: Noise generator & p\pageref{eofi2noise}\\
\hline
3*** & Message is coming from outside ("`onion packet"')) &\\
\hline
\end{longtable}
The command fields 2 and 3 are defined by the respective subsystem.
% -----------------------------------------------------------------------------
\subsection{Identification string (id)}
\label{idn}
To identify a packet, each packet contains an identification string,
which is \emph{EOF\_L\_ID} bytes long. It may contain only the
following characters:
\begin{itemize}
\item A-Z (alphabet in upper case)
\item a-z (alphabet in lower case)
\item 0-9 (the digits)
\item ! (exclamation mark)
\item - (minus)
\end{itemize}
The EOFs or EOFi may chose freely any of the \emph{68719476736}
possibilities.\footnote{$(26+26+10+2)^6$, as long as EOF\_L\_ID is 6.}
The characters are limited to those characters to allow easy debugging
and to keep the non-binary command layout.
% -----------------------------------------------------------------------------
\subsection{Size (size)}
\index{eof.h (File)}%
All sizes used in this document are "`symbolic sizes"': The real size
is defined in the attached file "`\emph{eof.h}"'.
Developers are advised to use the symbolic name in their programs.

A size is is always represented as an ASCII number found in a
fixed length string of \emph{EOF\_L\_SIZE} bytes.
% Nico: 1.0
% -----------------------------------------------------------------------------
\subsection{Nick name (nick)}
The peer name is a \emph{EOF\_L\_NICKNAME} byte fixed length string.
It is only used internally to give a peer a rememberable name ("`a nick'").
It is never transmitted over the network.
% Nico: 1.0
% -----------------------------------------------------------------------------
\subsection{Group name (group)}
The group name is a \emph{EOF\_L\_GROUP} byte fixed length string.
% Nico: 1.0
% -----------------------------------------------------------------------------
\subsection{Message text (msgtext)}
The message text is a \emph{EOF\_L\_MESSAGE} byte fixed length string.
% Nico: 1.0
% -----------------------------------------------------------------------------
\subsection{Peer address (addr)}
The address of a peer, which is is a \emph{EOF\_L\_ADDRESS}
byte fixed length string. Peer addresses are specified as
URLs as defined in RFC3986\cite{uri-1}. For more information have
a look at section \ref{tp} on page \pageref{tp}.
% Nico: 1.0
% -----------------------------------------------------------------------------
\subsection{Keyid, the fingerprint (keyid)}
A (PGP) fingerprint\footnote{See RFC 2440, 11.2. Key IDs and Fingerprints}
is a \emph{EOF\_L\_KEYID} byte fixed length string.
It does not contain any spaces.
It can be retrieved by issuing the following gpg-command:
\begin{verbatim}
LC_ALL=C gpg --fingerprint  | \
   grep "Key fingerprint =" | \
   sed -e 's/.*=//' -e 's/ //g' 
\end{verbatim}
% Nico: 1.0

