\section{Simple Data Types ("`EOFsdt"')}
The following sections define the simple datatypes.
They are further referenced as "`EOFsdt"'.
% Nico: 1.0
% -----------------------------------------------------------------------------
\subsection{Command (command)}
A command is represented as an ASCII number in a fixed length string of
4 bytes. It is used to identify the intent of a message.
\subsubsection{Examples}
\begin{itemize}
\item 1100
\item 3000
\item 2200
\end{itemize}
% -----------------------------------------------------------------------------
\subsection{Identification string (id)}
\label{eofid}
To identify a message, a message may contain an identification string,
called the \textit{EOFID}.
This ID is an integer that is encoded based on the following characters:
\begin{itemize}
\item A-Z (alphabet in upper case)
\item a-z (alphabet in lower case)
\item 0-9 (the digits)
\item ! (exclamation mark)
\item - (minus)
\end{itemize}
The order of the characters is as follows:

\small{\{0123456789abcdefghijklmnopqrstuvwxyzABCDEFGHIJKLMNOPQRSTUVWXYZ-!\}}.\\
The length of an EOFID is 6 bytes, which results in
\emph{68719476736} possible ids.\footnote{$(10+26+26+2)^6$}.
The given characters where selected to allow easy debugging.
% -----------------------------------------------------------------------------
\subsubsection{Examples}
The following examples en- and decode integers into the specified format.
Use is made of the Python reference implementation:
\begin{verbatim}
>>> import ceof
>>> ceof.EOFID.int_to_id(42)
'00000G'
>>> ceof.EOFID.int_to_id(1)
'000001'
>>> ceof.EOFID.int_to_id(64)
'000010'
>>> ceof.EOFID.id_to_int('000010')
64
>>> ceof.EOFID.id_to_int('!!!!!!')
68719476735
>>> ceof.EOFID.id_to_int('a-----')
11794116542
>>> ceof.EOFID.id_to_int('000000')
0
\end{verbatim}
% Nico: 1.0
% -----------------------------------------------------------------------------
\subsection{Size (size)}
A size is represented as an ASCII number in a fixed length
string of 6 bytes.
\subsubsection{Examples}
\begin{verbatim}
>>> import ceof
>>> ceof.fillup("10", 6)
'10\x00\x00\x00\x00'
>>> ceof.fillup("10000", 6)
'10000\x00'
>>> ceof.fillup("100000", 6)
'100000'
\end{verbatim}
% Nico: 1.0
% -----------------------------------------------------------------------------
\subsection{Peer name (name)}
The peer name is a 128 byte fixed length string.
\subsubsection{Examples}
\begin{verbatim}
>>> import ceof
>>> ceof.fillup("Thomas Hü", 128)
'Thomas Hü\x00\x00\x00\x00\x00\x00\x00\x00\x00\x00\x00\x00\x00\x00\x00\x00\x00\x00
\x00\x00\x00\x00\x00\x00\x00\x00\x00\x00\x00\x00\x00\x00\x00\x00\x00\x00\x00\x00
\x00\x00\x00\x00\x00\x00\x00\x00\x00\x00\x00\x00\x00\x00\x00\x00\x00\x00\x00\x00
\x00\x00\x00\x00\x00\x00\x00\x00\x00\x00\x00\x00\x00\x00\x00\x00\x00\x00\x00\x00
\x00\x00\x00\x00\x00\x00\x00\x00\x00\x00\x00\x00\x00\x00\x00\x00\x00\x00\x00\x00
\x00\x00\x00\x00\x00\x00\x00\x00\x00\x00\x00\x00\x00\x00\x00\x00\x00\x00\x00\x00\x00'
\end{verbatim}
% Nico: 1.0
% -----------------------------------------------------------------------------
\subsection{Group name (group)}
The group name is a 128 byte fixed length string. It is currently not being
used but reserved for future support of multi user chat.
\subsubsection{Examples}
\begin{verbatim}
>>> import ceof
>>> ceof.fillup("!eof", 128)
'!eof\x00\x00\x00\x00\x00\x00\x00\x00\x00\x00\x00\x00\x00\x00\x00\x00\x00\x00\x00\x00
\x00\x00\x00\x00\x00\x00\x00\x00\x00\x00\x00\x00\x00\x00\x00\x00\x00\x00\x00\x00\x00
\x00\x00\x00\x00\x00\x00\x00\x00\x00\x00\x00\x00\x00\x00\x00\x00\x00\x00\x00\x00\x00
\x00\x00\x00\x00\x00\x00\x00\x00\x00\x00\x00\x00\x00\x00\x00\x00\x00\x00\x00\x00\x00
\x00\x00\x00\x00\x00\x00\x00\x00\x00\x00\x00\x00\x00\x00\x00\x00\x00\x00\x00\x00\x00
\x00\x00\x00\x00\x00\x00\x00\x00\x00\x00\x00\x00\x00\x00\x00\x00\x00\x00\x00\x00'
\end{verbatim}
% Nico: 1.0
% -----------------------------------------------------------------------------
\subsection{Message text (msgtxt)}
The message text is a 256 byte fixed length string.
\subsubsection{Examples}
\begin{verbatim}
>>> import ceof
>>> ceof.fillup("Hello, !eof!", 256)
'Hello, !eof!\x00\x00\x00\x00\x00\x00\x00\x00\x00\x00\x00\x00\x00\x00\x00\x00\x00\x00
\x00\x00\x00\x00\x00\x00\x00\x00\x00\x00\x00\x00\x00\x00\x00\x00\x00\x00\x00\x00\x00
\x00\x00\x00\x00\x00\x00\x00\x00\x00\x00\x00\x00\x00\x00\x00\x00\x00\x00\x00\x00\x00
\x00\x00\x00\x00\x00\x00\x00\x00\x00\x00\x00\x00\x00\x00\x00\x00\x00\x00\x00\x00\x00
\x00\x00\x00\x00\x00\x00\x00\x00\x00\x00\x00\x00\x00\x00\x00\x00\x00\x00\x00\x00\x00
\x00\x00\x00\x00\x00\x00\x00\x00\x00\x00\x00\x00\x00\x00\x00\x00\x00\x00\x00\x00\x00
\x00\x00\x00\x00\x00\x00\x00\x00\x00\x00\x00\x00\x00\x00\x00\x00\x00\x00\x00\x00\x00
\x00\x00\x00\x00\x00\x00\x00\x00\x00\x00\x00\x00\x00\x00\x00\x00\x00\x00\x00\x00\x00
\x00\x00\x00\x00\x00\x00\x00\x00\x00\x00\x00\x00\x00\x00\x00\x00\x00\x00\x00\x00\x00
\x00\x00\x00\x00\x00\x00\x00\x00\x00\x00\x00\x00\x00\x00\x00\x00\x00\x00\x00\x00\x00
\x00\x00\x00\x00\x00\x00\x00\x00\x00\x00\x00\x00\x00\x00\x00\x00\x00\x00\x00\x00\x00
\x00\x00\x00\x00\x00\x00\x00\x00\x00\x00\x00\x00\x00\x00\x00\x00'
\end{verbatim}
% Nico: 1.0
% -----------------------------------------------------------------------------
\subsection{Peer address (address)}
The address of a peer, which is a 128 byte fixed length string. 
Peer addresses are specified as URLs as defined in RFC3986\cite{rfc3986}. 
\subsubsection{Examples}
\begin{verbatim}
>>> import ceof
>>> ceof.fillup("tcp://127.0.0.1:6667", 128)
'tcp://127.0.0.1:6667\x00\x00\x00\x00\x00\x00\x00\x00\x00\x00\x00\x00\x00\x00\x00
\x00\x00\x00\x00\x00\x00\x00\x00\x00\x00\x00\x00\x00\x00\x00\x00\x00\x00\x00\x00\x00
\x00\x00\x00\x00\x00\x00\x00\x00\x00\x00\x00\x00\x00\x00\x00\x00\x00\x00\x00\x00\x00
\x00\x00\x00\x00\x00\x00\x00\x00\x00\x00\x00\x00\x00\x00\x00\x00\x00\x00\x00\x00\x00
\x00\x00\x00\x00\x00\x00\x00\x00\x00\x00\x00\x00\x00\x00\x00\x00\x00\x00\x00\x00\x00
\x00\x00\x00\x00\x00\x00\x00\x00\x00'
\end{verbatim}
% Nico: 1.0
% -----------------------------------------------------------------------------
\subsection{Peer fingerprint (keyid)}
A (PGP) fingerprint\footnote{See RFC 2440\cite{rfc2440}, 11.2. Key IDs and Fingerprints}
is a 40 byte fixed length string. As the fingerprint has a fix length of
40 bytes, there is never padding needed.
\subsubsection{Examples}
\begin{verbatim}
% gpg --fingerprint | grep "Key fingerprint =" | sed -e 's/.*=//' -e 's/ //g'
A35767A98CA9CC3CE368679AB679548202C9B17D
\end{verbatim}
