\subsection{EOF commands and command fields (mapping table)}
An EOF command is exactly \emph{EOF\_L\_CMD} bytes long (fixed length string)
and contains an ASCII number.

EOF commands are the main method of communication between EOFs and EOFi.

The command field 0 indicates the direction.
The command field 1 indicates the EOF subsystem.
\begin{longtable}{|c|c|c|}
\caption{Command fields}\\
\hline
\textbf{Value} & \textbf{Subsystem} / \textbf{Description} & \textbf{Ref}\\
\hline
1*** & Message is coming from the EOF implementation &\\
\hline
10** & \textbf{eofi2tp}: Transport protocols & p\pageref{eofi2tp}\\
\hline
11** & \textbf{eofi2ui}: User interface & p\pageref{eofi2ui}\\
\hline
12** & \textbf{eofi2crypto}: Crypto engine & p\pageref{eofi2crypto}\\
\hline
13** & \textbf{eofi2noise}: Noise generator & p\pageref{eofi2noise}\\
\hline
2*** & Message is coming from EOF subsystem (internally) &\\
\hline
20** & \textbf{eofi2tp}: Transport protocols & p\pageref{eofi2tp}\\
\hline
21** & \textbf{eofi2ui}: User interface & p\pageref{eofi2ui}\\
\hline
22** & \textbf{eofi2crypto}: Crypto engine & p\pageref{eofi2crypto}\\
\hline
23** & \textbf{eofi2noise}: Noise generator & p\pageref{eofi2noise}\\
\hline
3*** & Message is coming from outside ("`onion packet"')) &\\
\hline
\end{longtable}
The command fields 2 and 3 are defined by the respective subsystem.

