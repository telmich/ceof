% ----------------------------------------------------------------------------
\section{Transport Protocols}
This chat protocol does not rely on a specific underlying transport protocol,
but is \textit{transport protocol independent}. Transport protocols are used
to submit 


On the "`wire"'. Different transports. Constant transport.
Define name (postcard?!) here. Includes transport specific
header / meta information.

% ----------------------------------------------------------------------------
% -----------------------------------------------------------------------------
\subsection{Network packets ("`postcards"')}
\label{postcards}
Onions are afterwards encapsulated into the transport protocol specific
packet type and sent out onto the network.
Postcards are described in detail in chapter \ref{postcards}, page \pageref{postcards}.

A postcard "`packet"' contains one onion packet plus the transport protocol
shell.  Postcard packets are the only packet type that is seen by a possible
attacker.  The name postcard was choosen to reflect the fact, that anyone
passing the postcard can read what is written on it.

All packets must be signed by the sender and encrypted for the receiver.
The different datatypes are just concatenated in the order described.
The following description of the content describes the pakckets
in their unencrypted form.
% Nico: 1.0

on-the-wire


The EOF messages described in the previous section are multiple times
encrypted and assembled according to the calculated source route.

These packets are codenamed "`postcards"', as it is assumed they can be read
by an attacker.
% ----------------------------------------------------------------------------
\subsection{Tunneling}
\label{tunneling}
aginst firewall rules / blocking of traffic
through http
steganographic

% ----------------------------------------------------------------------------
\subsection{Multiplexing / Variable Addresses}
\label{multiplexing}
Different transports for one peer.

% ----------------------------------------------------------------------------
\subsection{List of Supported Transports}
To ensure interoperability, clients which support a specific
protocol version must support all listed transport protocols.
\begin{longtable}{|c|c|c|}
\caption{Transport protocols}\\
\hline
\textbf{Protocol} & \textbf{Description} & \textbf{Supported versions}\\
\hline
tcp & Transmission Control Protocol & 0.1 - 0.1\\
\hline
\end{longtable}


