% #############################################################################
\section{Transport protocols ("`eofi2tp"')}
\label{eofi2tp}
% -----------------------------------------------------------------------------
\subsection{Definition/Specification}
URL, URI, Scheme clearification, use tcp:// and http:// syntax.


% -----------------------------------------------------------------------------
\subsection{Introduction}
Every transport protocol consists of a listening (\verb=listen=)
and sending (\verb=send=) part. When the EOFi starts, it starts all
transport protocols, which should the remain started and listen
for commands issued by the EOFi.
URLs are used as defined in RFC3986\cite{uri-1}.
For schemes only lower-case names are allowed
(to prevent problems with case-insensitive filesystems).
% -----------------------------------------------------------------------------
\subsection{Paths}
\label{tppaths}
All paths are relative to the directory p\_tp\_d, as described
in \ref{ptpd} on p. \pageref{ptpd}.
\begin{longtable}{|c|c|}
\caption{Required transport protocols}\\
\hline
\textbf{Scheme} & \textbf{Name}\\
\hline
\verb=available/= & Available transport procols\\
\hline
\verb=available/<scheme>/listen= & Listener (executable) for the scheme\\
\hline
\verb=available/<scheme>/send= & Sender (executable) for the scheme\\
\hline
\verb=listen/= & Enabled listeners\\
\hline
\verb=listen/<freeform>/url= & URL \\
\hline
\end{longtable}
% -----------------------------------------------------------------------------
\subsubsection{Enabling a scheme}
\label{tpscheme}
To enable support for the scheme \verb=http=
create both, the \verb=listen= and the \verb=send= executable below
\verb=available/http=.
% Nico: 1.0
% -----------------------------------------------------------------------------
\subsubsection{Enabling a listener}
\label{tplisten}
To enable a listener for the scheme \emph{http} at the URL
\emph{http://www.example.com/eof}, create
the directory \verb=listen/http-example= and the file
\verb=listen/http-example/url= with the content
\verb=http://www.example.com/eof=.
If the transport protocol implementation needs or allows additional
configuration files, they should be placed in the same directory.
The EOF implementation will parse the URL and check whether the
scheme is supported.
% Nico: 1.0
% -----------------------------------------------------------------------------
\subsection{Required implementations}
\label{tprequired}
This version of the EOF standard requires support for the following
transport protocols in the implementation:
\begin{longtable}{|c|c|c|c|}
\caption{Required transport protocols}\\
\hline
\textbf{Scheme} & \textbf{Name} & \textbf{Format} & \textbf{Description}\\
\hline
tcp & IP/TCP & \verb=tcp:host:port= & Plain tcp\\
\hline
\end{longtable}
% -----------------------------------------------------------------------------
\subsection{Upcoming implementations}
\label{tpupcoming}
The following transport protocol implementations have been proposed,
but are not yet implemented and thus not yet part of the standard:

\begin{longtable}{|c|c|c|c|}
\caption{Upcoming transport protocols}\\
\hline
\textbf{Scheme} & \textbf{Name} & \textbf{Format} & \textbf{Description}\\
\hline
dns & DNS       & \verb=dns:host:port:type= & Package via DNS\\
\hline
http & HTTP       & \verb=http:host:port/path= & Plain http\\
\hline
https & HTTPS     & \verb=https:host:port/path= & http with ssl\\
\hline
mailto & E-Mail   & \verb=mailo:address= & Send package as e-mail\\
\hline
mediawiki & Mediawiki   & \verb=mediawiki:host:port:page= & Transfer via Mediawiki\\
\hline
smb  & SMB     & \verb=smb:[user[:password]@]host:path= & Server message block\\
\hline
smtp & SMTP     & \verb=smtp:[user[:password]@]host:port= & http with ssl\\
\hline
tcps & IP/TCP/SSL & \verb=tcps:host:port= & tcp with ssl around\\
\hline
udp & IP/UPD      & \verb=udp:host:port= & Plain UDP\\
\hline
\end{longtable}
% -----------------------------------------------------------------------------
\subsection{Developing and integrating implementations}
If you want to create a new transport protocol, create the two
executables as described in \ref{tppaths} and put them in their
scheme directory.
If the implementation works fine and you would like to add it to EOF,
create the directory \verb=tp/<scheme>/<implementation_name>= within
the EOFi source directory and update the following
sections of this document:
\begin{itemize}
\item \ref{tprequired}
\item \ref{tprequired} / new subsubsection for the new scheme
\item \ref{tpupcoming}
\end{itemize}

The name of the subsubsection is the name of the scheme
plus the registered implementation ("`like tcp-c"').
Every such subsubsection must at least contain:
\begin{itemize}
\item Author contact information
\item URL of website
\item Programming language
\end{itemize}
Keep the subsubsections sorted by alphabet.
% Nico: 1.0
% -----------------------------------------------------------------------------
\subsection{1000: Send packet}
\index{Command!1000}
\begin{longtable}{|c|c|c|c|}
\caption{Command 1000 parameters}\\
\hline
\textbf{Parameter} & \textbf{Type} & \textbf{Description} & \textbf{Example}\\
\hline
ID & EOFsdt: id & packet id & afdb12\\
\hline
Destination & EOFsdt: addr & complete URL (with "`\emph{scheme:}"') & tcp:127.0.0.3:42\\
\hline
Size & EOFsdt: size & Size of message, excluding this header & 424242\\
\hline
Message & Binary data & The message & BLOB\\
\hline
\end{longtable}
\subsubsection{Possible answers}
\begin{itemize}
\item 2000
\item 2001
\end{itemize}
% -----------------------------------------------------------------------------
\subsubsection{1000: Example}
Added linebreak after some \textbackslash{}0 for readability, which are \textbf{not} in
the real packet!
\begin{verbatim}
1000abfudh127.0.0.3:42\0\0\0\0\0\0\0\0\0\0\0\0\0\0\0\0\0\0\0\0\0\0\0\0\0\0\0\0
\0\0\0\0\0\0\0\0\0\0\0\0\0\0\0\0\0\0\0\0\0\0\0\0\0\0\0\0\0\0\0\0\0\0\0\0\0\0\0
\0\0\0\0\0\0\0\0\0\0\0\0\0\0\0\0\0\0\0\0\0\0\0\0\0\0\0\0\0\0\0\0\0\0\0\0\0\0\0
\0\0\0\0\0\0\0\0\0\010\0\0\0\0HEREISDATA
\end{verbatim}
% Nico: 1.0
% -----------------------------------------------------------------------------
\subsection{1001: Enable listening}
\index{Command!1001}%
When the listening transport protocol starts up, EOFi sends this command and
waits for the acknowledge package, before it marks the transport protocol as enabled.
\begin{longtable}{|c|c|c|c|}
\caption{Command 1001 parameters}\\
\hline
\textbf{Parameter} & \textbf{Type} & \textbf{Description} & \textbf{Example}\\
\hline
ID & EOFsdt: id & packet id & afdb12\\
\hline
Destination & EOFsdt: addr & URL without "`\emph{scheme:}"' & 127.0.0.3:42\\
\hline
\end{longtable}
\subsubsection{Possible answers}
\begin{itemize}
\item 2003
\end{itemize}
% -----------------------------------------------------------------------------
\subsubsection{1001: Example}
\begin{verbatim}
1001afdb12127.0.0.3:42\0\0\0\0\0\0\0\0\0\0\0\0\0\0\0\0\0\0\0\0\0\0\0\0\0
\0\0\0\0\0\0\0\0\0\0\0\0\0\0\0\0\0\0\0\0\0\0\0\0\0\0\0\0\0\0\0\0\0\0\0\0
\0\0\0\0\0\0\0\0\0\0\0\0\0\0\0\0\0\0\0\0\0\0\0\0\0\0\0\0\0\0\0\0\0\0\0\0
\0\0\0\0\0\0\0\0\0\0\0\0\0\0\0\0\0\0\0
\end{verbatim}
% -----------------------------------------------------------------------------
\subsection{1002: Stop listening}
\index{Command!1002}%
This command requests a listening transport protocol to shutdown.
It should free all ressources and exit. After a grace time (maybe seconds,
not yet defined), EOFi will kill the badly behaving transport protocol.
\begin{longtable}{|c|c|c|c|}
\caption{Command 1001 parameters}\\
\hline
\textbf{Parameter} & \textbf{Type} & \textbf{Description} & \textbf{Example}\\
\hline
ID & EOFsdt: id & packet id & afdb12\\
\hline
\end{longtable}
\subsubsection{Possible answers}
\begin{itemize}
\item none
\end{itemize}
% -----------------------------------------------------------------------------
\subsection{1003: Prepare sending}
\index{Command!1003}%
This command requests a transport protocol sender to prepare for sending out
data. The transport protocol should initialise itself and acknowledge
when it is ready for sending.
\begin{longtable}{|c|c|c|c|}
\caption{Command 1003 parameters}\\
\hline
\textbf{Parameter} & \textbf{Type} & \textbf{Description} & \textbf{Example}\\
\hline
ID & EOFsdt: id & packet id & afdb12\\
\hline
\end{longtable}
\subsubsection{Possible answers}
\begin{itemize}
\item none
\end{itemize}
% -----------------------------------------------------------------------------
\subsection{1004: Stop sending}
\index{Command!1004}%
This command requests a transport protocol sender to shutdown.
It should free all ressources and exit. After a grace time (maybe seconds,
not yet defined), EOFi will kill the badly behaving transport protocol.
\begin{longtable}{|c|c|c|c|}
\caption{Command 1004 parameters}\\
\hline
\textbf{Parameter} & \textbf{Type} & \textbf{Description} & \textbf{Example}\\
\hline
ID & EOFsdt: id & packet id & afdb12\\
\hline
\end{longtable}
\subsubsection{Possible answers}
\begin{itemize}
\item none
\end{itemize}
% -----------------------------------------------------------------------------
\subsection{2000: Packet successfully sent}
This code is returned by the transport protocol subsystem to EOFi on success.
After this return code, the transport protocol exits.
\begin{longtable}{|c|c|c|c|}
\caption{Command 2000 parameters}\\
\hline
\textbf{Parameter} & \textbf{Type} & \textbf{Description} & \textbf{Example}\\
\hline
ID & EOFsdt: id & packet id & afdb12\\
\hline
\end{longtable}
\subsubsection{Possible answers}
\begin{itemize}
\item none
\end{itemize}
% -----------------------------------------------------------------------------
\subsection{2001: Packet not sent}
\index{Command!2001}
This code is returned by the transport protocol subsystem to the
EOF implementation on failure.
After this return code, the transport protocol exits.
\begin{longtable}{|c|c|c|c|}
\caption{Command 2001 parameters}\\
\hline
\textbf{Parameter} & \textbf{Type} & \textbf{Description} & \textbf{Example}\\
\hline
ID & EOFsdt: id & packet id & afdb12\\
\hline
\end{longtable}
\subsubsection{Possible answers}
\begin{itemize}
\item none
\end{itemize}
% -----------------------------------------------------------------------------
\subsection{2002: Received packet}
\index{Command!2002}
The listening transport protocol received a packet and notifies EOFi.
\begin{longtable}{|c|c|c|c|}
\caption{Command 2002 parameters}\\
\hline
\textbf{Parameter} & \textbf{Type} & \textbf{Description} & \textbf{Example}\\
\hline
ID & EOFsdt: id & packet id & afdb12\\
\hline
Size & EOFsdt:size & Excluding the 2002 and this field & 42\\
\hline
Message & Binary data & The message & BLOB\\
\hline
\end{longtable}
\subsubsection{Possible answers}
\begin{itemize}
\item none
\end{itemize}
% -----------------------------------------------------------------------------
\subsection{2003: Listening}
\index{Command!2003}
This code is returned by the listening transport protocol, as soon as the
listening process is ready to receive data. After that, EOFi can announce
the listening URL to other EOFi.
\begin{longtable}{|c|c|c|c|}
\caption{Command 2003 parameters}\\
\hline
\textbf{Parameter} & \textbf{Type} & \textbf{Description} & \textbf{Example}\\
\hline
ID & EOFsdt: id & packet id & afdb12\\
\hline
\end{longtable}
\subsubsection{Possible answers}
\begin{itemize}
\item none
\end{itemize}
% -----------------------------------------------------------------------------
\subsection{2004: Not listening}
\index{Command!2004}
If something happended that makes the listening transport protocol
unable to startup, it issues this command and exits afterwards.
\begin{longtable}{|c|c|c|c|}
\caption{Command 2004 parameters}\\
\hline
\textbf{Parameter} & \textbf{Type} & \textbf{Description} & \textbf{Example}\\
\hline
ID & EOFsdt: id & packet id & afdb12\\
\hline
Reason & EOFsdt: msgtxt & Why the connection was refused & Too many UIs connected.\\
\hline
\end{longtable}
\subsubsection{Possible answers}
\begin{itemize}
\item none
\end{itemize}
% -----------------------------------------------------------------------------
\subsection{2005: Ready for sending}
\index{Command!2005}
This code is returned by the transport protocol sender, as soon as the
process is ready to send data.
\begin{longtable}{|c|c|c|c|}
\caption{Command 2005 parameters}\\
\hline
\textbf{Parameter} & \textbf{Type} & \textbf{Description} & \textbf{Example}\\
\hline
ID & EOFsdt: id & packet id & afdb12\\
\hline
\end{longtable}
\subsubsection{Possible answers}
\begin{itemize}
\item none
\end{itemize}
% -----------------------------------------------------------------------------
\subsection{2006: Sender preperation error}
\index{Command!2006}
If something happended that makes the transport protocol sender
unable to startup, it issues this command and exits afterwards.
\begin{longtable}{|c|c|c|c|}
\caption{Command 2006 parameters}\\
\hline
\textbf{Parameter} & \textbf{Type} & \textbf{Description} & \textbf{Example}\\
\hline
ID & EOFsdt: id & packet id & afdb12\\
\hline
Reason & EOFsdt: msgtxt & Why the connection was refused & Too many UIs connected.\\
\hline
\end{longtable}
\subsubsection{Possible answers}
\begin{itemize}
\item none
\end{itemize}
% #############################################################################
% #############################################################################


