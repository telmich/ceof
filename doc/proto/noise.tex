\section{Noise}

purpose of ... 

There are many situations in which an EOFi sends out data to the network,
although you did not write a message: In fact, as EOFi \textbf{always}
sends packets in a fixed interval, it needs to have data to encrypt and send.

Noise can be any type of random data. As the current random number generators
are quite expensive, it is recommend to use a huge dictionary, old 
messages, logfiles, public emails, etc. for noise input.
% Nico: 1.0

% ----------------------------------------------------------------------------
\subsection{Fixed interval for sending}
To prevent de-anonymisation by doing statistical analysis on the traffic,
postcards are being at a fixed rate.
In case there is not message to be send, messages with no receiving
party (\textit{noise}) must be used instead.

To avoid problems with different sending intervals
(queuing, buffer exceeding) a network wide fixed sending interval
should be chosen. From the calculations made above, an interval
of \textbf{0.25s} seems to provide a good tradeoff between
latency, anonymity and bandwidth usage.

The number of proxy peers can be choosen by every peer individually
and can be used to focus on smaller bandwidth usage or higher degree
of anonymity. A value of 5 in a network of 100 peers
would make de-anonymising less probable than winning Lotto in Germany
(6 of 49). A value of 8 delays the message at maximum 2 seconds until
it is received.

% ----------------------------------------------------------------------------
\subsubsection{Average Latency}
The average latency for a peer is calculated as follows:
$$\frac{\sum\limits_{i=0}^{(Proxy peers +1)}}{Proxy peers + 1}$$
Depending on how long the waiting delay in seconds is, this results
in different average waiting times for message to arrive, as expressed in
the following formula:
$$Delaytime * \frac{\sum\limits_{i=0}^{(Proxy peers +1)}}{Proxy peers + 1}$$

\begin{longtable}{|c|c|c|c|c|c|c|}
\caption{Average latency based on number of proxy peers}
\label{avglatencypeers}\\
\hline
\multirow{2}{*}{\textbf{Proxy peers}} & \multicolumn{6}{|l|}{\textbf{Average latency (multiplied by delay times)}} \\
& \textbf{(\# timeslots)} & \textbf{0.125s} & \textbf{0.25s} & \textbf{0.5s} & \textbf{1s} & \textbf{2s}\\
\hline
\textbf{1} & 0.5 & 0.0625s & 0.125s & 0.25s & 0.5s & 1s\\
\hline
\textbf{2} & 1 & 0.125s & 0.25s & 0.5s & 1s & 2s\\
\hline
\textbf{3} & 1.5 & 0.1875s & 0.375s & 0.75s & 1.5s & 3s\\
\hline
\textbf{4} & 2 & 0.25s & 0.5s & 1s & 2s & 4s\\
\hline
\textbf{5} & 2.5 & 0.3125s & 0.625s & 1.25s & 2.5s & 5s\\
\hline
\textbf{6} & 3 & 0.375s & 0.75s & 1.5s & 3s & 6s\\
\hline
\textbf{7} & 3.5 & 0.4375s & 0.875s & 1.75s & 3.5s & 7s\\
\hline
\textbf{8} & 4 & 0.5s & 1s & 2s & 4s & 8s\\
\hline
\textbf{9} & 4.5 & 0.5625s & 1.125s & 2.25s & 4.5s & 9s\\
\hline
\textbf{10} & 5 & 0.625s & 1.25s & 2.5s & 5s & 10s\\
\hline
\end{longtable}
A study about 
\textit{System Response Time and User Satisfaction}\cite{responsetime}
indicates that response times up to 6 seconds are tolerable, these numbers
cannot be related directly to the tolerated delay, because the receiving
peer does not know when the sending peer sent a message.
% ----------------------------------------------------------------------------
\subsubsection{Maximum Latency}
Furthermore besides the average latency, the maximum latency needs
to be taken into account, which is shown in table \ref{maxlatencypeers}.
\begin{longtable}{|c|c|c|c|c|c|c|}
\caption{Maximum latency based on number of proxy peers}
\label{maxlatencypeers}\\
\hline
\multirow{2}{*}{\textbf{Proxy peers}} & \multicolumn{6}{|l|}{\textbf{Maximum latency (multiplied by delay times)}} \\
& \textbf{(\# timeslots)} & \textbf{0.125s} & \textbf{0.25s} & \textbf{0.5s} & \textbf{1s} & \textbf{2s}\\
\hline
\textbf{1} & 1 & 0.125s & 0.25s & 0.5s & 1s & 2s\\
\hline
\textbf{2} & 2 & 0.25s & 0.5s & 1s & 2s & 4s\\
\hline
\textbf{3} & 3 & 0.375s & 0.75s & 1.5s & 3s & 6s\\
\hline
\textbf{4} & 4 & 0.5s & 1s & 2s & 4s & 8s\\
\hline
\textbf{5} & 5 & 0.625s & 1.25s & 2.5s & 5s & 10s\\
\hline
\textbf{6} & 6 & 0.75s & 1.5s & 3s & 6s & 12s\\
\hline
\textbf{7} & 7 & 0.875s & 1.75s & 3.5s & 7s & 14s\\
\hline
\textbf{8} & 8 & 1s & 2s & 4s & 8s & 16s\\
\hline
\textbf{9} & 9 & 1.125s & 2.25s & 4.5s & 9s & 18s\\
\hline
\textbf{10} & 10 & 1.25s & 2.5s & 5s & 10s & 20s\\
\hline
\end{longtable}
% ----------------------------------------------------------------------------
\subsection{Bandwidth usage}
Based on the calculated packet sizes (\ref{packetsizes}), 
the following \textbf{outgoing bandwidth}
is needed, as can be seen in table \ref{bandwidth}.
The highest needed bandwidth
in case of 10 proxy peers sending at an interval of 0.125s results in 64 KiB/s
or 512 KBit/s. Todays DSL connections usually exceed this limit by magnitudes
\begin{longtable}{|c|c|c|c|c|c|}
\caption{Outgoing bandwidth usage}
\label{bandwidth}\\
\hline
\multirow{2}{*}{\textbf{Proxy peers}} & \multicolumn{5}{|l|}{\textbf{Intervals / Bandwidth usage in KiB/s}} \\
& \textbf{0.125s} & \textbf{0.25s} & \textbf{0.5s} & \textbf{1s} & \textbf{2s}\\
\hline
\textbf{1} & 9.6 & 4.8 & 2.4 & 1.2 & 0.6\\
\hline
\textbf{2} & 15.2 & 7.6 & 3.8 & 1.9 & 0.95\\
\hline
\textbf{3} & 20.8 & 10.4 & 5.2 & 2.6 & 1.3\\
\hline
\textbf{4} & 26.4 & 13.2 & 6.6 & 3.3 & 1.65\\
\hline
\textbf{5} & 32 & 16 & 8 & 4 & 2\\
\hline
\textbf{6} & 38.4 & 19.2 & 9.6 & 4.8 & 2.4\\
\hline
\textbf{7} & 44.8 & 22.4 & 11.2 & 5.6 & 2.8\\
\hline
\textbf{8} & 50.4 & 25.2 & 12.6 & 6.3 & 3.15\\
\hline
\textbf{9} & 56.8 & 28.4 & 14.2 & 7.1 & 3.55\\
\hline
\textbf{10} & 64 & 32 & 16 & 8 & 4\\
\hline
\end{longtable}
Even on mobile networks, HSUPA Category 1 delivers 0.73 Mbit/s upstream bandwidth.
Older ISDN and Modem connections would not suffice, though
today there are a lot of alternative connections available.\cite{wiki:bitrates}
If using only 9 proxy servers with 0.125s delay or
10 proxy servers with a delay of 0.25s, EDGE could be supported.

The total required network bandwidth is the product of peers in the
network multiplied by the outgoing bandwidth per peer:
$$Required Network Bandwidth = Network Peers * Peer Bandwidth$$
Thus a network with 100 peers, all of them which use 10 proxy peers
and send at a rate of 0.125s, the total required network bandwidth would be
$$RNB = 100 * 64 KiB/s = 6400 KiB/s = 6.25MiB/s = 50 MBit/s$$
If also the receiving side is inside the same network, the required network
bandwith has to be multiplied by two and thus results in 
\textit{100 MBit/s}. This is exactly half of what a fast ethernet switch
can deliver, because fast ethernet provides full duplex 100Mbit/s streams
and thus a single fast ethernet fabric would support 200 simultaneous users.
As the network bandwidth cannot be controlled or influenced by the user
directly, these calculations may simply indicate the network bandwidth
usage for network providers. The user can focus on the bandwith specifications
found in table \ref{bandwidth}.


