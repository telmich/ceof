\section{Onion Routing}
\label{onionrouting}
Onion routing is used to ensure sender-receiver anonymity. In contrast
to the Tor-network, there are no exit nodes and the receiver can be anyone
in the onion chain (see fig. \ref{onionrouting}).

% ----------------------------------------------------------------------------
\subsection{Onions}
\label{eofonion}

They are building the base for the EOF protocol.
Onions are described in detail in chapter \ref{onions}, page \pageref{onions}.

An onion packet is a (multiple times) encrypted packet.
An onion packet contains at least one plaintext packet, but can also contain
already encrypted packets. It may look like as follows:

All data that is transferred over the network must be encrypted.


All packets must be signed by the sender and encrypted for the receiver.
The different datatypes are just concatenated in the order described.
The following description of the content describes the pakckets
in their unencrypted form.
% Nico: 1.0


The EOF messages described in the previous section are multiple times
encrypted and assembled according to the calculated source route.

% ----------------------------------------------------------------------------
\subsection{Variable Peers}
Different routes for every packet

% ----------------------------------------------------------------------------
\subsection{Source based routing}
Either here or in Intra Machine Intra Client
% ----------------------------------------------------------------------------
\subsection{Peer selection}
Which peers, how many. Constant? May give upper bounds of latency.

8 * 0.5seconds = 4 seconds delay.


%% \begin{figure}
%%     \centering
%%     \caption{Onion Routing}
%%     \label{onionrouting}
%%     \includegraphics[scale=0.8]{onionrouting.eps}
%% \end{figure}
% ----------------------------------------------------------------------------
\subsection{Packet sizes}
The average packet size depending on the number of peers it was
(re-)encrypted for can be found in table \ref{pkgsizes}.
It was generated by running the reference implementation
(\verb=ceof onion -m "test" peer0  | wc -c=)
and increasing the number of proxy peers to be inserted.
The resulting size includes only the final onion,
but does not include transport protocol headers.
\begin{longtable}{|c|c|}
\caption{Packet sizes (experimental)}
\label{pkgsizes}\\
\hline
\textbf{Proxy peers} & \textbf{Packet size (in KiBiBytes)}\\
\hline
\textbf{1} & 1.2\\
\hline
\textbf{2} & 1.9\\
\hline
\textbf{3} & 2.6\\
\hline
\textbf{4} & 3.3\\
\hline
\textbf{5} & 4.0\\
\hline
\textbf{6} & 4.8\\
\hline
\textbf{7} & 5.6\\
\hline
\textbf{8} & 6.3\\
\hline
\textbf{9} & 7.1\\
\hline
\textbf{10} & 8.0\\
\hline
\end{longtable}



