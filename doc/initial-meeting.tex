% first presentation about cmtp
\pdfminorversion=4
%\documentclass[ucs]{beamer}
\documentclass{beamer}
%\documentclass[utf8]{beamer}
\usepackage[utf8]{inputenc}
\usepackage{german}
\usepackage{graphicx}
\usepackage{beamerthemesplit}
\setbeamercovered{dynamic}
\usetheme{Malmoe}
\usecolortheme{crane}

\title{Development of a secure, decentralised anonymous chat system}
\subtitle{Initial Meeting}

\author{Nico -telmich- Schottelius}

\date{??. ?? 2012}

\begin{document}
\frame{\titlepage}

%\section[Outline]{}
\frame{\tableofcontents}

\section{Initial Situation}

\frame
{
  \frametitle{Initial Situation}
  \begin{itemize}
  \item Various non-secure chat systems available (silc, irc, ...)
  \item Commercial, encrypted solution (skype)
  \item Anonymous Routing (Tor)
  \item No usable decentralised, anonymous secure chat system available
  \end{itemize}
}

\section{Motivation}
\frame
{
  \frametitle{Motivation}
  \begin{itemize}
     \item Fix the situation
     \item Close gap
     \item Chat, geschlossenens netz (keine exit nodes)
     \item Proof of concept
  \end{itemize}
}

\section{Project}
\frame
{
  \frametitle{Projektphases}
  \begin{itemize}
     \item Current state analysis
     \item Protocol design
     \item Implementation of Prototype
  \end{itemize}
}

\frame
{
  \frametitle{Project Steps}
  \begin{enumerate}
     \item Detailed analysis and comparison of legacy chat systems
        to summarise current chat system features and their
        security characteristics.
    \item Analysis of features and security requirements
    \item Analysis of related communication protocols
    \item Development of a new chat protocol
    \item Development of chat prototype using the new chat protocol
    \item Preparation of a live demonstration of the prototype
  \end{enumerate}
}

\frame
{
  \frametitle{Expected Results}
  \begin{itemize}
    \item Report and comparison of current chat systems including strength and weaknesses
    \item Requirement analysis
    \item Report of related communication protocols including strength and weaknesses
    \item Protocol definition paper (containing chat features,
        data types, transport methods, security measures)
    \item Implementation of a prototype for the new chat system
    \item Presentation of a successful anonymous, decentralised chat session, which
       includes proof of the required security by using example attacks.
  \end{itemize}
}


\frame
{
  \frametitle{Questions?}
  \begin{center}
  ?
  \end{center}
}


%\frame
%{
%  \frametitle{Funktionalität zusammenfassen}
%  \begin{itemize}[<+->]
%     \item "`Typen"' (types)
%     \item conf/type/*
%     \item \_\_ vor jedem Namen (Shell-Umgebung)
%     \item z.B. Netzseite, Mailserver, Wiki, ...
%  \end{itemize}
%}

%\begin{frame}[fragile]
%   \frametitle{Ein neuer Typ}
%   \begin{small}
%   \begin{verbatim}
%   mkdir conf/type/__my_mailserver
%   cat << eof > conf/type/__my_mailserver/manifest
%   __package nullmailer --state installed
%
%   require="__package/nullmailer" \
%       __file /etc/nullmailer/remotes \
%          --source "$__type/files/remotes"
%   eof
%   chmod u+x conf/type/__my_mailserver/manifest
%
%   mkdir conf/type/__my_mailserver/files
%   echo my.fancy.smart.host > \
%       conf/type/__my_mailserver/files/remotes
%   \end{verbatim}
%   \end{small}
%\end{frame}

%\section{Aktualisieren}
%\begin{frame}[fragile]
%  \frametitle{Versionen}
%  \begin{itemize}[<+->]
%     \item x.y: Stabile Version
%     \item master: Entwicklung
%  \end{itemize}
%\end{frame}

%\begin{frame}[fragile]
%  \frametitle{Aktualisieren}
%  \begin{center}
%  git pull
%  \end{center}
%\end{frame}


\end{document}
