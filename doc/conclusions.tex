\chapter{Conclusions}
The given objectives as defined in section
\ref{objectives} have been reached. In the analysis of chat systems
the bad feeling regarding Skype proved true. Information
about the internals of the implementation and the protocol disappeared
from the Internet during the development stage of this thesis
and government agencies claim to be able to wire tap
communication via Skype. Therefore the need for a secure anonymous chat system
has been reconfirmed.
The analysis of the related communications protocols showed that research
for anonymity systems has been a long ongoing topic pursued by both
academics and technology enthusiasts (hackers).
The definition of the chat protocol and the 
development of the prototype implementation were tightly
coupled together by use of the spiral software development model.

Summarised it is a great feeling to see that the theoretical ideas described
in the chat protocol as well as the practical implementation have been realised.
As far as I can see, this thesis produced a working prototype of a secure,
decentralised and anonymous chat system. 
It is able to hide who is talking to whom.
% ------------------------------------------------------------------------------
\section{Review}
The analysis of existing anonymity systems generally revealed the
following aspects:
\begin{itemize}
\item Anonymity systems have been built for a long time
\item Anonymity systems often take a conservative approach regarding resource usage
\item Anonymity systems are usually built as an abstract network, not for a specific application
\end{itemize}
%% The two latter topics are what differentiates this thesis from the older
%% approaches: Constantly sending data out is not tried to be avoided, but
%% actually used to improve the degree of anonymity twice:
My thesis deviates from other approaches with regard to the latter two topics.
Instead of avoiding to constantly send out data, this technique is actually
used to improve the degree of anonymity. It does so in two ways:
\begin{enumerate}
\item Avoid statistical packet analysis
\item More traffic means more effort needed for potential attackers
\end{enumerate}
However, it has to be acknowledged that there are situations
in which this approach may not be desirable.
For instance in a situation when high volume traffic is expensive
(like on some mobile device contracts).
Furthermore, due to the constant sending, it is easy to detect and probably
impossible to hide that someone is participating in this chat network.
With the use of transport protocol multiplexing and transport protocol
tunnelling it is possible to enhance the availability, but in case an
institution forbids the use of this chat protocol, it can reliably
detect people using it and terminate the complete network link.

Developing the anonymity system specific for an application has the advantage
of ease of use, but the disadvantage of knowledge and resource duplication
for the implementation. On the other hand, this work
chat could be used as a basis for expansion: taken the
defined protocol and its prototype, it is possible to generalise
the protocol to support more features and to enhance the prototype.
% ------------------------------------------------------------------------------
\section{Future Work}
Although the defined chat protocol is in a usable
state and the proof-of-concept implementation works, there are further
enhancements that are to be considered for future work.
% ------------------------------------------------------------------------------
\subsection{Prototype Enhancements}
% ------------------------------------------------------------------------------
\subsubsection{Authenticity Checks}
The current prototype does not support checking authenticity: It only checks
whether the message can be decrypted, but does not check whether the message
has been signed and whether the signature is correct.
% ------------------------------------------------------------------------------
\subsubsection{Verify Cross OS Support}
The prototype has been tested on Linux only. As the implementation does not
contain any known OS specific calls, it should also work on other operating
systems, but this has to be verified.
% ------------------------------------------------------------------------------
\subsection{Transport Protocol Support}
The current prototype only supports TCP for sending and receiving data.
To effectively support transport protocol multiplexing, the number of supported
protocols should be increased. Table \ref{newtps} shows a proposal of some
transport protocols, which could be supported.
\begin{longtable}{|c|c|c|c|}
\caption{Upcoming transport protocols}
\label{newtps}\\
\hline
\textbf{Scheme} & \textbf{Name} & \textbf{Format} & \textbf{Description}\\
\hline
dns & DNS       & \verb=dns://[host:port]/domain/type= & Domain Name Service\\
\hline
http & HTTP       & \verb=http://host:port/resource= & Hypertext Transfer Protocol\\
\hline
https & HTTPS     & \verb=https://host:port/resource= & HTTP encrypted with SSL\\
\hline
mailto & E-Mail   & \verb=mailto://address= & Send message via e-mail\\
\hline
mediawiki & Mediawiki   & \verb=mediawiki://host:port/page= & Communication via Mediawiki\\
\hline
smb  & SMB     & \verb=smb://[user[:password]@]host/path= & Server Message Block\\
\hline
smtp & SMTP     & \verb=smtp://[user[:password]@]host:port= & Simple Mail Transfer Protocol\\
\hline
tcps & IP/TCP/SSL & \verb=tcps//host:port= & TCP encrypted with SSL\\
\hline
udp & IP/UPD      & \verb=udp://host:port= & Plain UDP\\
\hline
\end{longtable}
In case a transport protocol is used for indirect traffic 
(see section \ref{accessmethods}),
two separate addresses need to be specified, as can be seen
in an example for e-mail:
\begin{enumerate}
\item a public address for announcing: (mailto://nico@example.org)
\item a private address for polling (imap://nico@example.org:password@imapserver.example.org)
\end{enumerate}
% ------------------------------------------------------------------------------
\subsection{Impact of decreasing packet size}
The current approach does not take care of message size decreasing on its
path due to removal of onion layers. This could potentially allow an attacker
to gain knowledge about the communication. Thus the implications of missing
padding remain to be analysed.
% ------------------------------------------------------------------------------
\subsection{Adding Peers / Key Exchange}
\label{keyexchange}
Currently all peers are added manually by the user. 
This process is tedious and prevents the network from growing quickly.
There are several possibilities to support automatic finding and
adding peers:
% ------------------------------------------------------------------------------
\subsubsection{Peer Exchange Messages}
\label{peerexchange}
The existing set of messages (see \ref{eofmsg}) could be expanded to a message
type that allows to send peer information (name, address, keyid) to another
peer. Furthermore a new message type could be introduced to request a random
peer from another peer, which may or may not include the public key 
and/or one or all addresses of the random peer as well.
% ------------------------------------------------------------------------------
\subsubsection{PGP Key Servers}
PGP keys are usually stored on public key servers.
If all public keys that are used for the chat network have a special comment
in them, like \textit{EOF42KEY}, it is easy to search for participants of the
network. After a peer selected a random key from a keyserver with the
special comment, it can contact it by using the given e-mail address.
The contacted peer can then reply with a \textit{peer exchange message}
(see \ref{peerexchange}) to submit all of its addresses.
% ------------------------------------------------------------------------------
\subsubsection{Key Signing Parties}
There are so called Key Signing Parties, at which PGP users exchange and
sign their public keys, after identification verification.
These parties could be upgraded to \textit{peer exchange and key signing
parties}.
% ------------------------------------------------------------------------------
\subsection{Using Trust Levels}
As the underlying technology, PGP, supports trust levels, one could use this
technique to determine from which peers to accept real (non noise)
messages to prevent spamming.
% ------------------------------------------------------------------------------
\subsection{Multi User Chat}
As most chat systems support multi user chat, this may be a feature to be added
in a future version of the protocol. The previously described trust levels
could be used to decide which people are allowed to join or send messages
to a specific group.
% ------------------------------------------------------------------------------
\subsection{Spread Usage}
As can be seen in section \ref{degreeofanonymity} on page \pageref{degreeofanonymity},
the degree of anonymity highly depends on the number of nodes in the network.
Therefore to enhance anonymity, a certain set of nodes needs to be established,
before a high degree of anonymity can be expected.
To spread usage, semi or fully automatic key exchanges, as described
in section \ref{keyexchange}, could be used. Additionally 
games or screensaver could be built, which are participating in the
chat network and support it by noise sending.
