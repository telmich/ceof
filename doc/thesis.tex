\documentclass[12pt,a4paper]{article}
\usepackage[utf8]{inputenc}
\usepackage{url}

\usepackage{a4wide}
\usepackage{longtable}           % lange Tabellen
%\usepackage[dvips]{epsfig}

\usepackage{fancyhdr}
\pagestyle{fancy}
%\fancyhf{}                              % bisherige Kopf- und Fusszeilen loeschen
\fancyhead[R]{Nico Schottelius}            % rechter Kopfzeileneintrag
%\fancyhead[L]{Projekdokumentation}      % linker Kopfzeileneintrag
%\fancyhead[L]{\thepage}      % linker Kopfzeileneintrag
%\fancyfoot[C]{\thepage}                % Fusszeileneintrag (Seitenzahl zentriert)
\renewcommand{\headrulewidth}{0.4pt}  % Strichstaerke unter der Kopfzeile
% ----------------------------------------------------------------------------
% let's start
\begin{document}
\title{Development of a secure, peer-to-peer, decentralised anonymous chat system}
\date{\today}
\author{Nico Schottelius (nico-hsz-t (o) schottelius.org)}
% ----------------------------------------------------------------------------
\maketitle
\newpage
% ----------------------------------------------------------------------------
% Inhaltsverzeichnis
\tableofcontents
\listoftables
%\listoffigures
\newpage

% ----------------------------------------------------------------------------
\section{Introduction}
\subsection{Abstract}
hier oder weiter oben? - siehe lisa vortrag!

\subsection{Abbreviations}
\subsection{Starting Position}
local, skype, untrustworthy

\subsection{Motivation}
From EBS + EOF

\subsection{Objectives}
From EOF + Latency "`low"', eventual consistency

% ----------------------------------------------------------------------------
\section{Related work/Alternatives/Existing solutions}
\subsection{IRC+SSL}
\subsection{SILC}
\subsection{Onion Routing}
\subsection{Tor}
\subsection{Comparison}
All the solutions with objectives.

% ----------------------------------------------------------------------------
\section{Design (which? better name)}
The theory behind EOF.
Search for documents proving/supporting the idea.

Different programs handle different objectives
\subsection{Use objectives and derive}
\subsection{Additional constraints}
Cross-OS. Posix and/or ansi-c. ui changable.
\subsection{Key Exchange}
Not implemented
% ----------------------------------------------------------------------------
\section{Intra Machine Intra Program Protocol}
\subsection{Interfaces}
Sockets, Environment, Paths, etc.
\subsection{Data Types}
\subsection{Sub Programs}
Consists of the following sub-programs:
encryption, dictionary, key/peer exchange daemon (if possible in this work),
% ----------------------------------------------------------------------------
\section{Sub Programs}
Describe what it does, how it does and where it is implemented.
\subsection{Noise / Dictionary / Database}
Generate input for times when there is no user input.
Random or db or or or.
Networt traffic
\subsection{Encryption}
Splitting of encryption into a seperate program can make use of
multiple computing ressources.
\subsection{User Interface}
Own program, indepentend of core.
\subsection{Transports}
Receive/Send

% ----------------------------------------------------------------------------
\section{Inter Machine Protocol}
On the "`wire"'. Different transports. Constant transport.

\subsection{Bandwidth usage}
\begin{verbatim}
>>> messages_size=4*1024
>>> messages_per_second=4
>>> bytes_per_second=messages_per_second*messages_size
>>> print(bytes_per_second)
16384
>>> bytes_per_day=86400*bytes_per_second
>>> print(bytes_per_day)
1415577600
>>> print(bytes_per_day/1024**2)
1350.0
>>> bytes_per_month=month_length*bytes_per_day
>>> print(bytes_per_month/1024**2)
41062.5
>>> print(bytes_per_month/1024**3)
40.10009765625
\end{verbatim}
16 KiB/s or 128 KBit/s, 2 ISDN lines. Around 1.4 GiB per day or
circa 40 GiB per month.

% ----------------------------------------------------------------------------
\section{Inter Machine Inter Program protocol}
After decoding the received packet. Forward, etc.
Based on part os EOF simple data types.

% ----------------------------------------------------------------------------
\section{Implementation}
This section describes the actual implementation
\subsection{Cross OS Support}
Non-unix socket!

%\input{titel.tex}                   % done
%\input{erklaerung.tex}              % done
% ----------------------------------------------------------------------------
%\input{fazit/ausblick.tex}          % done
% ----------------------------------------------------------------------------
% ----------------------------------------------------------------------------
% Anhang
\appendix

\section{References}
\begin{thebibliography}{666}
\bibitem{otr} \url{http://www.cypherpunks.ca/otr/}
\bibitem{tor} Tor project, \url{http://www.torproject.org/},
\url{http://www.icir.org/vern/cs294-28/papers/dingledine.pdf}
\bibitem{tor2} Low-Cost Traffic Analysis of Tor,
    \url{http://www.cl.cam.ac.uk/~sjm217/papers/oakland05torta.pdf}
\bibitem{untrace} Untraceable electronic mail, return addresses, and digital pseudonyms,
\url{http://citeseerx.ist.psu.edu/viewdoc/download?doi=10.1.1.79.7468&rep=rep1&type=pdf}
\bibitem{onion} Onion Routing, \url{http://www.onion-router.net/}
\bibitem{python-gpg} Various implementations for Python,
    \url{http://wiki.python.org/moin/GnuPrivacyGuard}
\bibitem{python-gnupg} python-gnupg - A Python wrapper for GnuPG,
    \url{http://packages.python.org/python-gnupg/}

\bibitem{unused} \url{http://www.privoxy.org/}
\url{https://yro.slashdot.org/story/11/10/08/0326235/russian-telco-mts-bans-skype-other-voip-services}

\url{http://en.wikipedia.org/wiki/Dining_cryptographers_protocol}
\url{http://crypto.stanford.edu/~pgolle/papers/nim.pdf}
\url{http://www.disappearing-inc.com/D/dcnetworks.html}

\end{thebibliography}

% ----------------------------------------------------------------------------
\end{document}
