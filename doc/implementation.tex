% ----------------------------------------------------------------------------
\chapter{Implementation of the Prototype}
The prototype was developed using spiral software development model.
It is named \textit{ceof}. The source code can be found in the subdirectory
\textit{src}.
% 5. Technical documentation and prototype for the new chat system
% ----------------------------------------------------------------------------
\section{Environment}
The prototype was developed using the Python3 programming language.
In addition to the standard libraries, the python-gnupg module is required.
To create a python environment that is suitable for running and developing
the prototype, a new virtualenv including the required python-gnupg module
can be created using the following commands:
\begin{verbatim}
% virtualenv -p /usr/bin/python3 python-env
% . ./python-env/bin/activate
% pip install python-gnupg
% (cd python-env/bin && ln -s python python3)
\end{verbatim}
Because python is an interpreted scripting language with interpreters
available for all major plattforms, the prototype should be runable
on all major plattforms.
All configurations are saved in the \textit{cconfig}\cite{cconfig} format.
The path to the configuration directory is usually derived by taking
the content of the environment variable \textit{HOME} and appending
the subdirectory \textit{.ceof}. On Linux this is referenced as
\textit{\textasciitilde{}/.ceof/}.
% ----------------------------------------------------------------------------
\section{Usage}
% ----------------------------------------------------------------------------
\subsection{Command Line Interface (CLI)}
The implemented prototype can be used using a command line interface.
All operations are grouped into commands, which are handled by the
executable \textit{bin/ceof}. If a subcommand is followed by the parameter
\textit{-h}, then a usage screen is displayed.
% ----------------------------------------------------------------------------
\subsection{Cryptographic Operations}
All cryptographic operations are accessable by using the \textit{crypto}
command, as shown in figure \ref{cryptohelp}.
\begin{figure}
\caption{Crypto Command Usage}
\label{cryptohelp}
\begin{verbatim}
(python-env)[18:40] brief:src% ./bin/ceof crypto -h
usage: ceof crypto [-h] [-d] [-v] [-c CONFIG_DIR]
                   [--encrypt ENCRYPT [ENCRYPT ...]] [--decrypt] [-e] [-f]
                   [-g] [-i] [-l LENGTH] [--name NAME]
                   [--email-address EMAIL_ADDRESS] [-s]

optional arguments:
  -h, --help            show this help message and exit
  -d, --debug           Set log level to debug
  -v, --verbose         Set log level to info, be more verbose
  -c CONFIG_DIR, --config-dir CONFIG_DIR
                        Select configuration directory ($HOME/.ceof by
                        default)
  --encrypt ENCRYPT [ENCRYPT ...]
                        Encrypt from stdin (specify recipients)
  --decrypt             Decrypt from stdin
  -e, --export          Export public key to stdout
  -f, --fingerprint     Show key fingerprint
  -g, --gen-key         Generate new private/public key pair
  -i, --import          Import public key from stdin
  -l LENGTH, --length LENGTH
                        Specify bit length for key generation
  --name NAME           Name (for key generate)
  --email-address EMAIL_ADDRESS
                        E-Mail-Address (for key generate)
  -s, --show            Show private/public key pair

Get ceof at http://www.nico.schottelius.org/software/ceof/
\end{verbatim}
\end{figure}
At first is it required to create a new public/private key pair using
the \textit{--gen-key} parameter (figure \ref{genkey}), which can
be displayed using the \textit{--show} parameter (figure \ref{showkey})
and exported using the \textit{--export}
parameter (figure \ref{exportkey}).
\begin{figure}
\caption{Generation of Public/Private Key Pair}
\label{genkey}
\begin{verbatim}
(python-env)[18:52] brief:src% ./bin/ceof crypto --gen-key 
    --name "Nico Schottelius" --email-address "nico@example.org"
\end{verbatim}
\end{figure}
\begin{figure}
\caption{Show Public/Private Key Pair}
\label{showkey}
\begin{verbatim}
(python-env)[18:54] brief:src% ./bin/ceof crypto --show
{'dummy': '', 'keyid': 'C5FC26760DD842D6', 'expires': '', 'length': '2048',
'ownertrust': '', 'algo': '1',
'fingerprint': '77E54EF64A6395FF2769B2F4C5FC26760DD842D6',
'date': '1339174371', 'trust': '', 'type': 'sec',
'uids': ['Nico Schottelius (EOF42KEY) <nico@example.org>']}
\end{verbatim}
\end{figure}
\begin{figure}
\caption{Export Public Key}
\label{exportkey}
\begin{verbatim}
(python-env)[19:23] brief:src% ./bin/ceof crypto --export
-----BEGIN PGP PUBLIC KEY BLOCK-----
Version: GnuPG v2.0.19 (GNU/Linux)

mQENBE/SLeMBCACu5sWt3j/ZTqZZ5eZw+cTvkIG6DwWaeVZjv+A+Dd7xZhbMBeyZ
q70CuOEURGLQUQQKtyT7bvTBjk8lkL2zcgIJ2a/MQQneJc+fEqB+ovlPM+Bl4qLf
TIuBMPnI+1OMOuTx0Agtys+6/9YaIdKaedtIqrZhVVsbaFAeE6MTHSm0i9bTtvyk
bH+X0JurCNL8nKEjf6SdSrQGdmohV/VyQTGlMZPaYG58LjCMKxbqWMb3lVKsmyRr
N4bZFPePzqBJzmqyH/noyoNuzbSUhNvUw27JzTL51u1JfMm2kQmkG1NZgLwXg6/W
e5FYbVoVI3LMj9NDABZ42y3mCv0QJJP6LkvdABEBAAG0Lk5pY28gU2Nob3R0ZWxp
dXMgKEVPRjQyS0VZKSA8bmljb0BleGFtcGxlLm9yZz6JATgEEwECACIFAk/SLeMC
Gy8GCwkIBwMCBhUIAgkKCwQWAgMBAh4BAheAAAoJEMX8JnYN2ELW8rcH/3Hdanzp
mUNfF7RqlU7sCwrGKFABOvTZtWQBfURLfTW2kMZRZEViRu6t3aB0hK3g1HWaSBzb
yXJmH6UznpkOG5gD+Y/FfMCiR7VZaiFXEZh9ukRWDwCytouoILdrey08Wr4YQEDf
+Ny38gLYu06Svnm25iQn3LiejTohCny5POkOnxfyVxOEhQ6LUjai6j0bSKk05o62
b2ZdKpGHsBqo9eHLr4y83JmoO5pSXDByBG0pu2Ukczey8BGuBwngUwEN/XKrl1xZ
aYHlpVNhCNsXthSAJdlag5Auju/t2S978yelO4Ii411dyDPrYZKjd4TGWbWfeVpS
jXOLmX++gs2tPyM=
=LB0W
-----END PGP PUBLIC KEY BLOCK-----
\end{verbatim}
\end{figure}
Importing keys is possible via standard input (\textit{stdin})
and show in figure \ref{importkey}.
\begin{figure}
\caption{Import Public Key}
\label{importkey}
\begin{verbatim}
# Export key from test peer0  and import into normal
# configuration directory
./bin/ceof crypto --config-dir ../test/peers/0 --export | 
    ./bin/ceof crypto --import
\end{verbatim}
\end{figure}
as well as the import
of public keys 
Furthermore de- and encryption are supported.
% -----------------------------------------------------------------------------
\subsection{Listener}
The \textit{listener} command is used
to configures to which addresses the prototype is listening to (figure
\ref{listenerusage}).
\begin{figure}
\caption{Listener Command Usage}
\label{listenerusage}
\begin{verbatim}
(python-env)[18:41] brief:src% ./bin/ceof listener -h
usage: ceof listener [-h] [-d] [-v] [-c CONFIG_DIR] [-a ADD] [-l] [-r REMOVE]

optional arguments:
  -h, --help            show this help message and exit
  -d, --debug           Set log level to debug
  -v, --verbose         Set log level to info, be more verbose
  -c CONFIG_DIR, --config-dir CONFIG_DIR
                        Select configuration directory ($HOME/.ceof by
                        default)
  -a ADD, --add ADD     Add an address to listen on
  -l, --list            List listener
  -r REMOVE, --remove REMOVE
                        Remove an address to listen on

Get ceof at http://www.nico.schottelius.org/software/ceof/
\end{verbatim}
\end{figure}
For the initial setup it is required to configure at least one listener
using the \textit{--add} parameter, which can be shown afterwards using
\textit{--list} parameter (figure \ref{addandlistlistener}). 
Protection against adding unsupported addreses is included.
\begin{figure}[htb]
\caption{Add and list listener addresses}
\label{addandlistlistener}
\begin{verbatim}
(python-env)[19:23] brief:src% ./bin/ceof listener --list
(python-env)[19:42] brief:src% ./bin/ceof listener --add tcp://0.0.0.0:42507
(python-env)[19:42] brief:src% ./bin/ceof listener --add tcp://0.0.0.0:42508
(python-env)[19:42] brief:src% ./bin/ceof listener --list                   
tcp://0.0.0.0:42507
tcp://0.0.0.0:42508
(python-env)[19:42] brief:src% ./bin/ceof listener --add foo://0.0.0.0:42508 
Unknown protocol in address foo://0.0.0.0:42508
\end{verbatim}
\end{figure}
% -----------------------------------------------------------------------------
\subsection{Noise}
The noise command does not accept any parameters and will output noise to
standard output (\textit{stdout}).
To be able to generate noise, the prototype requires UTF-8 encoded
files to be present in the noise directory (\textit{\textasciitilde{}/.ceof/noise}).
Possible sources of noise are the archive of RFCs or
the Linux Kernel sources, which are copied
into the noise directory as shown in figure \ref{initnoise}.
\begin{figure}
\caption{Init Noise Directory}
\label{initnoise}
\begin{verbatim}
% mkdir -p ~/.ceof/noise
% cd ~/.ceof/noise 
% find ~/linux/linus -name '*.c' -type f -exec cp {} . \;
% find ~/rfc/mirror/rfcs-text-only/ -name '*.txt' -type f -exec cp {} . \;
% ls | wc -l
23287
\end{verbatim}
\end{figure}
The noise command is only provided for debugging purposes.
% -----------------------------------------------------------------------------
\subsection{Peer}
The peer command is used to add, remove and list peers, as
well as adding and removing addresses to peers. Its usage
is shown in \ref{peerusage}.
\begin{figure}[htb]
\caption{Peer Command Usage}
\label{peerusage}
\begin{verbatim}
(python-env)[20:18] brief:src% ./bin/ceof peer -h
usage: ceof peer [-h] [-d] [-v] [-c CONFIG_DIR] [-a] [-r] [-l]
                 [-f FINGERPRINT] [--add-address ADD_ADDRESS]
                 [--remove-address REMOVE_ADDRESS]
                 [name]

positional arguments:
  name                  Name of the peer (myself: you)

optional arguments:
  -h, --help            show this help message and exit
  -d, --debug           Set log level to debug
  -v, --verbose         Set log level to info, be more verbose
  -c CONFIG_DIR, --config-dir CONFIG_DIR
                        Select configuration directory ($HOME/.ceof by
                        default)
  -a, --add             Add a peer
  -r, --remove          Remove a peer
  -l, --list            List peers
  -f FINGERPRINT, --fingerprint FINGERPRINT
                        Specify fingerprint for peer
  --add-address ADD_ADDRESS
                        Add an address to a peer
  --remove-address REMOVE_ADDRESS
                        Remove an address from a peer

Get ceof at http://www.nico.schottelius.org/software/ceof/
\end{verbatim}
\end{figure}
When adding a peer, it is required to import its public key using
the \textit{crypto} command before (see figure \ref{importkey}).
After adding a peer, addresses can be added or removed
(see figure \ref{peeraddcore}).
\begin{figure}[htb]
\caption{Peer Add}
\label{peeraddcore}
\begin{verbatim}
(python-env)[21:45] brief:src% ./bin/ceof peer --add peer0 
    --fingerprint 729BD24186E4E3F7EA3872FCAB29961528ACE126
(python-env)[21:46] brief:src% ./bin/ceof peer 
    --add-address tcp://127.0.0.1:42222 peer0                         
(python-env)[21:46] brief:src% ./bin/ceof peer --list                                    
peer0/729BD24186E4E3F7EA3872FCAB29961528ACE126/['tcp://127.0.0.1:42222']
(python-env)[21:47] brief:src% ./bin/ceof peer 
    --add-address tcp://127.0.0.1:42223 peer0 
(python-env)[21:47] brief:src% ./bin/ceof peer --list                                    
peer0/729BD24186E4E3F7EA3872FCAB29961528ACE126/['tcp://127.0.0.1:42222',
     'tcp://127.0.0.1:42223']
\end{verbatim}
\end{figure}
% -----------------------------------------------------------------------------
\subsection{Onion}
The onion command is used to create and send onions to other peers.
The help page is shown in figure \ref{onionusage}.
\begin{figure}[htb]
\caption{Onion Command Usage}
\label{onionusage}
\begin{verbatim}
(python-env)[20:16] brief:src% ./bin/ceof onion -h        
usage: ceof onion [-h] [-d] [-v] [-c CONFIG_DIR] [-m MESSAGE]
                  [-r REPEAT_COUNT] [-s]
                  [name]

positional arguments:
  name                  Name of the peer

optional arguments:
  -h, --help            show this help message and exit
  -d, --debug           Set log level to debug
  -v, --verbose         Set log level to info, be more verbose
  -c CONFIG_DIR, --config-dir CONFIG_DIR
                        Select configuration directory ($HOME/.ceof by
                        default)
  -m MESSAGE, --message MESSAGE
                        Create onion with this message for peer
  -r REPEAT_COUNT, --repeat-count REPEAT_COUNT
                        Repeat action n times (used for timing/profiling)
  -s, --send            Send message created to peer

Get ceof at http://www.nico.schottelius.org/software/ceof/
\end{verbatim}
\end{figure}
To create and display the onion, which would usually be send on the network,
use the \textit{\-\-message} parameter (figure \ref{onioncreate}. 
To actually send it, use the \textit{-{}-send} parameter.
\begin{figure}[htb]
\caption{Create and send an Onion}
\label{onioncreate}
\begin{verbatim}
(python-env)[21:57] brief:src% ./bin/ceof onion -m 'Hello dear peer0!' peer0
Onion chain: -----BEGIN PGP MESSAGE-----
Version: GnuPG v2.0.19 (GNU/Linux)

hQEMA0TOaZwAXdx8AQf/Uu8OGiVQRSTSW1ExuzFJpebZBHsHx7MdchDdF9QlfW1V
[...]
\end{verbatim}
\end{figure}
% -----------------------------------------------------------------------------
\subsection{Server}
The \textit{server} command is used to start
the server and does usually not take any arguments,
though specific servers parts can be disabled
(see figure \ref{serverusage}). Disabling the listener
makes it impossible to receive messages, disabling
noise prevents the server from sending messages regulary
and disabling the UI Server disables listening for chat
UIs (see chapter \ref{chatui}).
\begin{figure}[htb]
\caption{Server Command Usage}
\label{serverusage}
\begin{verbatim}
(python-env)[22:09] brief:src% ./bin/ceof server -h
usage: ceof server [-h] [-d] [-v] [-c CONFIG_DIR] [-l] [-n] [-u]
                   [-a UI_ADDRESS] [-p UI_PORT]

optional arguments:
  -h, --help            show this help message and exit
  -d, --debug           Set log level to debug
  -v, --verbose         Set log level to info, be more verbose
  -c CONFIG_DIR, --config-dir CONFIG_DIR
                        Select configuration directory ($HOME/.ceof by
                        default)
  -l, --no-listener     Disable listener server
  -n, --no-noise        Disable noise sending
  -u, --no-ui           Disable UI server
  -a UI_ADDRESS, --ui-address UI_ADDRESS
                        Listen on this address for UI connections
  -p UI_PORT, --ui-port UI_PORT
                        Listen on this port for UI connections

Get ceof at http://www.nico.schottelius.org/software/ceof/
(python-env)[22:09] brief:src% 
\end{verbatim}
\end{figure}
% -----------------------------------------------------------------------------
\subsection{Transport Protocols (TP)}
The \textit{tp} command is used to manage transport protocols,
its usage is shown in figure \ref{tpusage}.
The parameter \textit{-{}-list} shows the available protocols,
the parameter \textit{-{}-route} generates a random route to
a peer. Both commands are show in figure \ref{tplistroute}.
\begin{figure}[htb]
\caption{TP Command Usage}
\label{tpusage}
\begin{verbatim}
(python-env)[22:04] brief:src% ./bin/ceof tp -h      
usage: ceof tp [-h] [-d] [-v] [-c CONFIG_DIR] [--chain-to] [-l] [-r] [name]

positional arguments:
  name                  Name of the peer

optional arguments:
  -h, --help            show this help message and exit
  -d, --debug           Set log level to debug
  -v, --verbose         Set log level to info, be more verbose
  -c CONFIG_DIR, --config-dir CONFIG_DIR
                        Select configuration directory ($HOME/.ceof by
                        default)
  --chain-to            Generate onion package for given peer
  -l, --list            List available transport protocols
  -r, --route-to        Generate route to given peer

Get ceof at http://www.nico.schottelius.org/software/ceof/
\end{verbatim}
\end{figure}
\begin{figure}[htb]
\caption{TP List and Route}
\label{tplistroute}
\begin{verbatim}
(python-env)[22:05] brief:src% ./bin/ceof tp -l      
tcp
(python-env)[22:07] brief:src% ./bin/ceof tp --route peer0
[<peer9/9205979C4E7D5F5896520E2941324091BA8B832A>, 
<peer1/999C49CF1E9BBDAB0DE169324E1521D6E125E86C>, 
<peer0/729BD24186E4E3F7EA3872FCAB29961528ACE126>, 
<peer3/2FBD910F1C4C6032B399513B57647432847125F4>, 
<peer6/4A85FEED9DDAAE37EE68F5DC1E4208D4E5FD84EB>, 
<peer4/07D0FD80AF46E51DAEE4363144CE699C005DDC7C>]
(python-env)[22:07] brief:src% 
\end{verbatim}
\end{figure}
% ----------------------------------------------------------------------------
\section{Code Examples}
% ----------------------------------------------------------------------------
\subsection{Modular Design}
UI seperated.
Portable Core.
id, base 64, 
queuing
forks
manual poll

noise

\subsubsection{Sub Programs}
Consists of the following sub-programs:
encryption, dictionary, key/peer exchange daemon (if possible in this work),
modular design
Different programs handle different objectives

\subsection{Encryption}
Splitting of encryption into a seperate program can make use of
multiple computing ressources.
\subsection{Transport protocols}

% ----------------------------------------------------------------------------
\section{Test of the Prototype}
%    6. Description of test results

sniffing on the wire
regular packets (noise)
encryption
receiving peer can decrypt message



% ----------------------------------------------------------------------------
\subsection{Performance}
%\subsection{Packet size}
Re-testing of crypt (using test/recrypt): 5.2 - 5.3 KiB per packet.

\subsection{Bandwidth usage}
After protocol overhead...

% ----------------------------------------------------------------------------
\section{Features}
Based on the requirements defined in \ref{requirements}, 
p. \pageref{requirements}, the following features have been implemented. 
% ----------------------------------------------------------------------------
\subsection{Anonymity}
Multi-Hop onion routingg

Sender verification
Before the encrypt the packet, it is signed via public-key
cryptography\cite{pgp-1}. Thus only the receiver can verify the message sender.

\subsubsection{obfuscation}
- Hide message sending 

We don't think it's possible to hide that you are part of the chat network,
because some heuristics will be developed to detect the chat packets.
So we use a different idea:
Every participant of an EOF network will constantly send chat packets
with a pre-defined frequency (for instance every 250 ms). 
If you don't chat, \emph{noise} is sent.\footnote{Noise is just random
data, see below for a more detailled description of noise.}
The noise is also used to defend against timing analysis.
In case you are sending out a message, the message packet will be added to the
queue and sent within the next free time slot.

From outside it can easily be seen, that you are part of the network,
but not, if you sent a message.

\subsubsection{Hide message receiver}
The message packets are always sent indirectly via onion routing\cite{onion-1}.
The idea is taken from the Tor project\cite{tor-1}, though EOF uses an enhanced
version: EOF does not know about entry or exit nodes. If you are the intended
receiver you may or may not continue to forward the message, which is defined
by the sender of the message. That said, EOF must use source 
routing\cite{source-routing-1}.

To support onion routing, the sender of a packet needs to encrypt the packet
multiple times, once for each host that receives the packet. This may look
as follows:
\begin{enumerate}
\item Create message (from noise or user input)
\item Create source path
\item Create packet for last peer ("`pkg-last"')
\item Create packet for last-1 peer including \emph{pkg-last}
\item Continue until first peer is reached
\item Sent packet to first peer 
\end{enumerate}
Thus every peer only knows the previous and the next peer.
% Nico: 1.0


\subsection{Availability}
To prevent
successful elimination of the service, a decentralised architecture should


\subsubsection{Reliable against single user attacks}
Traditional chat networks depend on one or more central organised servers.
An attacker can stop all communication, if she runs a successful denial
of service ("`DoS"') attack against the central systems.
To protect against this, EOF uses a dynamic peer-to-peer network, which works
as long as the minimun number of peers and the destination peer is available.
It has no dependency on a central server.

\subsubsection{Hide packets in network stream}
As said before, we don't think it's possible to hide the participation in the
chat network. To be able to send packets, although an attacker \emph{knows}
about the participation, EOF embeds all chat packets into other (well known)
protocols (which is knows as steganography\cite{stegano-1}).
EOF does not implement nor specify \emph{transport protocols} itself.
The EOF community is urged to implement them in a creative way: Usage
of well-known protocols like TCP\cite{tcp-1}, HTTP\cite{http-1},
SMTP\cite{smtp-1} or even transmission of packets on avian
carriers\cite{avian-1} are encouraged. The tunneling of EOF packets through
those protocols (also know as obfuscation) makes it harder to detect
and \emph{block} EOF traffic. 
If an attacker wants you to stop sending messages, she has to completly
remove you from the network, because any open protocol may be (ab)used to
encapsulate EOF packets into it.
% Nico: 1.0 




\subsection{Confidentiality}

\subsection{Integrity}


% ----------------------------------------------------------------------------
\section{Related Project: Chat User Interface}
In addition to the described prototype, in a different
project a chat user interfaces has been
developed that can talk to the prototype using a chat server.
This chat UI is documented in the appendix (\ref{chatui}, p. \pageref{chatui}).
