% ----------------------------------------------------------------------------
\chapter{Introduction}

Anonymous communication networks were first intro-
duced by David Chaum in his seminal paper [10] describing
the mix as a fundamental building block for anonymity.
(aus tor2)

\section{Abstract}
hier oder weiter oben? - siehe lisa vortrag!

\section{Abbreviations}
\section{Starting Position}
local, skype, untrustworthy

\section{Motivation}
From EBS + EOF

\section{Objectives}
From EOF + Latency "`low"', eventual consistency

\subsection{Steganographic}
Not for hiding, but to support longer überleben?
http is pretty well suited for this.
mixing with smtp, imap, pop3, tcp. etc.

% -----------------------------------------------------------------------------
\section{History of the project}
The project started in 2007 as an idea of !eof\cite{!eof}
(a friendly hacker community). After several meetings it was clear that
we need to do some experiements and create phases to structure the
development.

!eof...
\subsection{The three phases}
So we divided the project into the three phases
\begin{enumerate}
\item \emph{EOF-1},
\item \emph{EOF-2},
\item \emph{EOF-3}.
\end{enumerate}
% -----------------------------------------------------------------------------
\subsubsection{EOF-1: Finding ideas}
The first phase was the so called "`finding ideas"' phase. We did some tests,
measured packet sizes and did some theorethic calculations on intervals.
There was also some discussion about implementing EOF in a ring structure,
like token ring\cite{token-ring}.
The first try to do the first implementation as a complet modularised
version began and stalled after some months of decentral developemnt.
% FIXME: cite
% Nico: 1.0
% -----------------------------------------------------------------------------
\subsubsection{EOF-2: Proof of concept}
We are currently working on a prototype in EOF-2. The idea is to get the
basic features working and to attract testers and developers.
% Nico: 1.0
% -----------------------------------------------------------------------------
\subsubsection{EOF-3: The final destination}
The idea of EOf-3 is to cleanup all parts of EOF-2 and create a
"`ready to be used"' software package, that may be used by end users.
This includes "`good documentation"' (this document).
% Nico: 1.0
% -----------------------------------------------------------------------------
\subsection{Further directions}
After the deployment of EOF-3, there may be more work necessary, like
\begin{itemize}
\item Deep analysis of security (by us and foreigners),
\item enhance performance,
\item port to other systems,
\item adapt to changed environment,
\item \ldots{}
\end{itemize}

% ----------------------------------------------------------------------------
\section{Codename}
This project is codename \textit{Eris Onion Forwarding ("`EOF"')}.
Onions because of architecture, Eris as the Goddess of Discordia to
prevent someone from controlling the chat.
