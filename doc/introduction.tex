% ----------------------------------------------------------------------------
\chapter{Introduction}
% -----------------------------------------------------------------------------
\section{Preamble}
This project originally started in 2007 as an idea of a the hacker
community \textit{!eof}, which has its roots in the the IRC
network.\footnote{Hacking here is referring to free and open source
software development as well as creative use of technology.}
Although various members of this community had written custom chat software
to chat "securely", most members, even of today, are using IRC as their
primary communication method.

Some virtual and physical meetings took place to overcome this weakness
and to develop a stronger, more appropriate protocol and software.
Due to the distributed nature of the community and low priorisation, not
much progress was made. As I am one of the members of this community,
this bachelor thesis and the support and interest
of the company I am working for, \textit{local.ch}, allows me to finish
this project.

The origan name for the project chosen was \textit{Eris Onion Forwarding}, or
short \textit{EOF}.\footnote{Eris the Goddess of Discordia as defined
in the PrincipiaDiscordia.}
I keep this name and thus several references and names
in this document are influenced by the original project name.
% -----------------------------------------------------------------------------
\section{Current status}
A lot of chat systems are currently available, 
but none of these systems implements an anonymous, decentralised and secure architecture.

Anonymity is important, when people need to hide to whom they are talking to.  
This is for instance true in extreme situations, when reporters located in 
dangerous areas, submit their messages in danger of life threats, but also 
applies to a lot of daily situations like sending confidential messages 
in untrusted networks: The basic knowledge of who is talking to whom may 
cause unwanted actions, for instance in politics or when being contacted by a head hunter.

A decentralised architecture helps to improve reliability as communication is
permitted to flow via different channels. This supports the ability to 
continue communications even under attack from an enemy. This is a general
requirement for reliable communications, but can be crucial for military in
case of war, as well as for reporters.

Secure chat provides the ability to hide message content from eavesdroppers,
as well as to ensure message and sender authenticity.  Without this technique
an attacker could use the information for harmful operations or change the
content of the messages. This threat is very dangerous i.e. in stock market 
situations, as well in the previously described situations.

The client, Thomas Gresch from local.ch, is running Skype for internal core 
communications. Skype is a closed source, encrypted, proprietary chat system 
owned by Microsoft. Because of the intensive efforts from Skype to hide how 
its internals are working, the software is considered untrustworthy.
The internal communication is a crucial part of daily business and an 
outage, whether caused by accident or intend, causes direct monetary 
loss for the company.

For this reason the research for an open alternative that may succeed 
the use of Skype is supported by local.ch.
% ----------------------------------------------------------------------------
\section{Objectives}
\label{objectives}
The objective of this thesis is the analysis, design and development of a secure, decentralised chat system. The main focus is on defining and documenting the mode of operation and the protocol definition. Additionally a prototype should be implemented. The thesis should be based on a detailled analysis of current chat systems. The analysis will not only consider chat systems, but also related secure and anonymous communication systems.
% ----------------------------------------------------------------------------
\section{Tasks}
\begin{enumerate}
\item Detailed analysis and comparison of open and legacy chat systems to summarise current chat system features and their security characteristics
\item Analysis of features and security requirements
\item Analysis of related communication protocols
\item Development of a new chat protocol
\item Development of chat prototype using the new chat protocol
\item Test of the prototype
\item Preparation of a live demonstration of the prototype
\end{enumerate}
% ----------------------------------------------------------------------------
\section{Expected Results}
\begin{enumerate}
\item Report and comparison of current chat systems including strength and weaknesses
\item Requirement analysis
\item Report of related communication protocols including strength and weaknesses
\item Protocol definition paper (containing chat features, data types, transport methods, security measures)
\item Technical documentation and prototype for the new chat system
\item Description of test results
\item Presentation of a successful anonymous, decentralised chat session, which includes proof of the required security by using example attacks.
\end{enumerate}
