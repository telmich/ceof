\chapter{Analysis of related communication protocols}
% ----------------------------------------------------------------------------
\section{Mix networks}
Chaum


Digital mixes (also known as mix networks) were invented by David Chaum in 1981
\cite{Chaum:1981:UEM:358549.358563}

Bilder auch auf \url{http://en.wikipedia.org/wiki/Mix_network}


% ----------------------------------------------------------------------------
\section{Onion Routing}
Patent \url{http://patft1.uspto.gov/netacgi/nph-Parser?Sect1=PTO2&Sect2=HITOFF&p=1&u=%2Fnetahtml%2FPTO%2Fsearch-bool.html&r=1&f=G&l=50&co1=AND&d=PTXT&s1=6266704.PN.&OS=PN/6266704&RS=PN/6266704}
% ----------------------------------------------------------------------------
\section{Tor}
On August 13, 2004 at the 13th USENIX Security Symposium,[4] Roger Dingledine, Nick Mathewson, and Paul Syverson presented Tor, The Second-Generation Onion Router.[5]

\url{http://en.wikipedia.org/wiki/Onion_routing}

Reply onions have been replaced by a rendezvous system, 


% ----------------------------------------------------------------------------
\section{I2P}
\url{http://en.wikipedia.org/wiki/I2P}



% ----------------------------------------------------------------------------
\section{OTR}
\url{http://de.wikipedia.org/wiki/Off-the-Record_Messaging}
% ----------------------------------------------------------------------------
\section{IP}

packet loss: ja
Packet duplicates:
Out of order packets: don't care - no fragmentation
    - probably missing conversation!
    Bit errors: not applicable
    Delay of packets: upper limit (?) 


    - verschieden kaäle:
        - email = sticky, cached, slow
            - tcp = fast, temporarily

            - rudp:
                http://www.javvin.com/protocolRUDP.html
                        rfc908, rfc1151

\subsection{TCP}
Reliable transport
\subsection{UDP}
Connectionless
Reliable connectionless\cite{rfc768}
% ----------------------------------------------------------------------------
\subsection{RUDP}
Reliable connectionless\cite{rfc908,rfc1151}

reliable: (up to a maximum number of retransmissions)
% ----------------------------------------------------------------------------
\section{Enet}

For unreliable packets, ENet will simply discard the lower sequence number packet if a packet with a higher sequence number has already been delivered. This allows the packets to be dispatched immediately as they arrive, and reduce latency of unreliable packets to an absolute minimum. For reliable packets, if a higher sequence number packet arrives, but the preceding packets in the sequence have not yet arrived, ENet will stall delivery of the higher sequence number packets until its predecessors have arrived.


Reliability

ENet provides optional reliability of packet delivery by ensuring the foreign host acknowledges receipt of all reliable packets. ENet will attempt to resend the packet up to a reasonable amount of times, if no acknowledgement of the packet's receipt happens within a specified timeout. Retry timeouts are progressive and become more lenient with every failed attempt to allow for temporary turbulence in network conditions.

Fragmentation and Reassembly

ENet will send and deliver packets regardless of size. Large packets are fragmented into many smaller packets of suitable size, and reassembled on the foreign host to recover the original packet for delivery. The process is entirely transparent to the developer.

Aggregation

ENet aggregates all protocol commands, including acknowledgements and packet transfer, into larger protocol packets to ensure the proper utilization of the connection and to limit the opportunities for packet loss that might otherwise result in further delivery latency.

Adaptability

ENet provides an in-flight data window for reliable packets to ensure connections are not overwhelmed by volumes of packets. It also provides a static bandwidth allocation mechanism to ensure the total volume of packets sent and received to a host don't exceed the host's capabilities. Further, ENet also provides a dynamic throttle that responds to deviations from normal network connections to rectify various types of network congestion by further limiting the volume of packets sent.

Portability

ENet works on Windows and any other Unix or Unix-like platform providing a BSD sockets interface. The library has a small and stable code base that can easily be extended to support other platforms and integrates easily. ENet makes no assumptions about the underlying platform's endianess or word size.

Freedom

ENet demands no royalties and doesn't carry a viral license that would restrict you in how you might use it in your programs. ENet is licensed under a short-and-sweet MIT-style license, which gives you the freedom to do anything you want with it (well, almost anything).

http://enet.bespin.org/
